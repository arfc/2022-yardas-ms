\chapter{Software Description}
Development on \SaltProc to enable use with \OpenMC consitutes a major portion
of this thesis. In this chapter, I will provide a high level overview
of my development process.\footnote{The release notes
contain more details for those interested} I will also provide an updated
description of the code structure using the new API. 

\section{SaltProc development history}%
\label{sub:saltproc-hisory}

\SaltProc\cite{rykhlevskii_saltproc_2018} is an open source Python package that
simulates on-line reprocessing via a batch-wise approach\footnote{Material is
moved to or from the core at specific time intervals} in liquid-fueled
\Gls{msr}s. More precisely, \SaltProc manages material flows and separation
processes on nuclides in the fuel. \SaltProc relies on external codes to simulate
fuel depletion.

The first version of \SaltProc (v0.1) was a simple Python 2.7 package that used
\SerpentTWO for the fuel depletion simulations. A single Python file contained
all functions; separation processes applie dto  used an implicit 100\%
efficiency \cite{rykhlevskii_advanced_2018}. The structure of \SaltProc v0.1 did
not lend itsef easily to development of more sophisticated treatments of
reprocessing. This led to the release of \SaltProc v0.2, in which the entire
codebase was refactored into an object oriented context in Python 3. The
addition of new functionality in the \verb.Process. classes enabled more
sophisticated treatment of online reprocessing \cite{rykhlevskii_fuel_2020} 

\SaltProc v0.3 saw the implementation of processes for gas sparging and
separation to more accurately simulate the \gls{msbr}, as well as additional
refactoring to better follow OOP concepts.

While v0.2 and v0.3 saw most of \SaltProc refactored for OOP, before I could
implement \OpenMC support, I needed to resolve the following issues:
\begin{enumerate}
    \item Relocate several functions to have better separation of concerns
    \item Overhaul the \SaltProc input file format giving users more control over their simulations
    \item Generalize docstrings\footnote{Since \SaltProc was initally written as a script wrapped around Serpent2, many of the docstrings explicitly referenced Serpent2.}
    \item Improve function and variable names
\end{enumerate}
I implemented these changes as part of the 0.4.0 release. The changes in that
release changed the API, making 0.4.0 incompatible with previous verions of \SaltProc.

\SaltProc v0.5.0 added full support for \OpenMC, overhauled the test suite, and included additional API changes.

% consider talking about automation?

\section{OpenMC}%
\label{sub:openmc}

OpenMC \cite{romano_jpenmc_2015} is an open source Monte Carlo particle
transport code. The \Gls{crpg} at \Gls{mit} started developing OpenMC back in
2011 with a focus on scalability for exascale computing. Since that time,
developers new and old contributed features (cite?) and fixes to the tool
expanding its scope and use cases. Notable features of OpenMC (as of version
0.13.1) are as follows \cite{homepage_openmc_2022}:
\begin{itemize}
    \item Support for fixed source, $k$-eigenvalue, and subcritical neutron multiplication cacluclations.
    \item Support for \Gls{csg} and \Gls{cad} geometry.
    \item Support for both continuous and multigroup transport calculations.
    \item Support for parallel execution via MPI and OpenMP.
    \item Geometry visualization through the Python API.
    \item Transport-coupled and transport-independent depletion
\end{itemize}

\section{SaltProc v0.5.0}
\label{sec:saltproc-detail}

\subsection{Material reprocessing}
Recall that \SaltProc uses a {\it batchwise} reprocessing scheme.
\ldots

\begin{equation}
    M^{\text{waste}} = \sum_{i} M^{i}_{\text{in}} * \epsilon_{i}
\end{equation}
\begin{equation}
    M^{\text{thru}} = \sum_{i} M^{i}_{\text{in}} * (1 - \epsilon_{i})
\end{equation}
\begin{equation}
    M^{i+1}_{q} = 
\end{equation}
Let $M_{i}$ be the mass of material $i$, $m_{i}$  be the mass flowrate of material $i$. 
