\chapter{Conclusions}
\label{ch:chapter6}

In this thesis, I have discussed an approximate \Gls{csg} model of the
\Gls{msbr} for use in \OpenMC. I have demonstrated this model to be consistent
with the \SerpentTWO model it is based on through a convergence study as well as
through demonstration of the results of the new \OpenMC functionailty in
\SaltProc in comparison with the \SerpentTWO functionality.
With the release of v0.5.0, I can happily say that \SaltProc is truly
open-source, as it no longer requires the use of export-controlled software
to function. Even so, I believe that there are several issues facing \SaltProc
that I want to bring attention to. Not all of these are problems with obvious
solutions, but they are still important to consider as they effect the long
term survival of the tool.

First, the need to reinitialize the geometry, neutronics
settings, and cross sections after applying the processeing scheme to the
materials is computationally inefficient. The most time consuming of these tasks
is loading cross sections, which can take up several node hours of additional
time if there are many timesteps. The solution to this problem is not obvious,
and will be depletion code-dependent.

Second, while in principle the tool is
designed to be compatible with any depletion code, in practice, this requires
implementing a new \verb.Depcode. class, which can potentially require major
refactoring as the work for this thesis did.

Third, as an interface code, \SaltProc needs to stay up-to-date with the
software it is intended for use with (\OpenMC and \SerpentTWO). If there are
major changes in the relevant parts of the API or results format, this change
will need to be addressed in a new \SaltProc release. While this is a normal
component of software development, right now there is only a single active
maintainer of \SaltProc (myself), and it is not a guarantee that I will
be able to continue to maintain it in the future.

The addition of continuous reprocessing functionality into \SerpentTWO -- and 
just recently, in  \OpenMC -- also call in to question the necessity for a
batch-wise reprocessing tool for advanced nuclear reactors. In countries using
LWRs that require fuel to be reprocessed (France and Japan, for example) perform
reprocessing, the reprocessing is physically batch-wise, happening once every 6
months or so (check this). I think for these kinds of simulations, \SaltProc
has a legitimate purpose for existing, as it is still the only open-source tool
of its kind. Assuming a suitable name-change, I think that it would work very
well as a tool for simulating MOX fuel in LWRs. I also think the graph-based
approach of material reprocessing has uses even for continuous reprocessing,
and could be added to other open source tools like \OpenMC.
