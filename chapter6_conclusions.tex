\chapter{Conclusions}
\label{ch:chapter6}
\glsresetall

In this thesis, I have discussed an approximate \Gls{csg} model of the
\Gls{msbr} for use in \OpenMC. I have demonstrated this model to be consistent
with the \SerpentTWO model through a convergence study as well as
through demonstration of the results of the new \OpenMC functionality in
\SaltProc in comparison with the \SerpentTWO functionality.
I found that the difference in $k_\text{eff}$ averaged 35 pcm
towards the beginning of the simulation, however this grew to
700 pcm towards the end of the simulation. I attributed this growth
to the difference in \ce{^{232}Th} between the \OpenMC and \SerpentTWO
calculations. I also found that actinides in general have very small
differences in mass between the two codes, around 2\%, and fission products
had even lower differences in mass, around 0.6\%. Finally, I attributed
large numerical discrepancies in the masses of \ce{^{241}Am}, \ce{^{242}Am},
\ce{^{242m}Am}, \ce{^{242}Cm}, \ce{^{245}Cm}, and \ce{^{246}Cm} present in the fuel
to an experimental \OpenMC feature I added to ensure accurate results for decay-only
nuclides, as the numerical discrepancies were only present at the beginning of the simulation.

\section{Future Work}
With the release of v0.5.0, \SaltProc has the capability to run
with a completely open source workflow. Even so, there are several issues facing \SaltProc
that I want to bring attention to. Not all of these are problems with obvious
solutions, but they are still important to consider as they affect the long
term survival of the tool.

First, the need to reinitialize the geometry, neutronics
settings, and cross sections after applying the processing scheme to the
materials is computationally inefficient. The most time consuming of these tasks
is loading cross sections, which can take up several node hours of additional
time if there are many timesteps. The solution to this issue depends on
the presence of an API, which will be depletion code-dependent.

Second, while in principle the tool is
designed to be compatible with any depletion code, in practice, this requires
implementing a new \verb.Depcode. class, which can potentially require major
refactoring as the work for this thesis did.

Third, as an interface code, \SaltProc needs to stay up-to-date with the
software it is intended for use with (\OpenMC and \SerpentTWO). If there are
major changes in the relevant parts of the API or results format, this change
will need to be addressed in a new \SaltProc release. While this is a normal
part of software development, right now there is only a single active
maintainer of \SaltProc (myself), and it is not a guarantee that I will
be able to continue to maintain it in the future.

The addition of continuous reprocessing functionality into \SerpentTWO -- and
recently, in  \OpenMC -- also calls in to question the necessity for a
batch-wise reprocessing tool for advanced nuclear reactors. In countries using
LWRs that require fuel to be reprocessed (France and Japan, for example) perform
reprocessing, the reprocessing is physically batch-wise, happening once every 6
months or so. In these use cases, \SaltProc is uniquely poised to simulate them,
as it is still the only open-source tool of its kind.
