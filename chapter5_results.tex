\chapter{Results}
\label{ch:chapter5}

\subsection{Platform and Software}
\label{sub:platform-software}
We ran our \OpenMC simulation on Theta, a supercomputer at ANL. Use of export
controlled code is currently prohibited on Theta, so for our \SerpentTWO runs,
we used a Dell Precision 3430 Workstation.

% make this subsection a table
% Talk about KNL, cross compiling, etc\ldots
Theta's compute nodes and constancia nodes used cross-compilation\ldots

\begin{table}[htpb] 
    \centering 
    \caption{Theta job parameters for 100k particle \OpenMC run}
    \label{tab:theta-params}
    \begin{tabular}{|c|c|c|} 
        \hline
        Quantity & \verb.aprun. option & Value\\
        \hline
        Nodes & \verb.-n. & 128 \\
        \hline
        MPI processes per node & \verb.-N. 4 \\
        \hline
        Threads per core & \verb.-j. & 4 \\
        \hline
        MPI Process binding & \verb.-cc. & depth \\
        \hline
        Depth & \verb.-d. & ?? \\
        \hline
        OpenMP threads per MPI process & \verb.-e OMP_NUM_THREADS=64. & 64 \\
        \hline
        Clustering mode & -- & SNC-4 \\
        \hline
        Memory mode & -- & Cache \\
        \hline
    \end{tabular}
\end{table}
% make this subsection a table

\subsection{Simulation Design and Parameters}
\label{sub:simulation-parameters}

I design


Table \ref{tab:saltproc-params} summarizes the simulation settings used
for both the \SerpentTWO and \OpenMC simulations.
 
\begin{table}[htpb] 
    \centering 
    \caption{Neutronics and Depletion parameters for SaltProc}
    \label{tab:saltproc-params}
    \begin{tabular}{|c|c|} 
        \hline
        Batches & 200 \\
        \hline
        Inactive batches & 80 \\
        \hline
        Particles per batch & 1e6 \\
        \hline
        Power [W] & 2.25e9 \\
        \hline
        Depletion steps & 122 \\
        \hline
        Depletion step length [days] & 3 \\
        \hline
        Depletion equation solver & IPF CRAM 48 \\
        \hline
        Time integration method & Euler's Method \\
        \hline
    \end{tabular}
\end{table}

\subsection{Data}
\label{sub:results-xs-data}

I used the ENDF/B-VII.1 library for both the \SerpentTWO and
\OpenMC simulations (cite endfb). I specifically used the neutron reaction,
neutron induced fission product yields, and spontaneous fission product yields
sublibraries. I used the thermal neutron scattering sublibrary from the
ENDF/B-VII.0 library, which contains the same data as what is in the
ENDF/B-VII.1 library. The thermal neutron scattering sublibray in ENDF/B-VII.1
uses a continuous representation that \SerpentTWO v2.1.32 does not support.

To ensure data consistency, I donwloaded the \verb,.ace, files from the NNDC
website, then processed these files into HDF5 format using the \OpenMC Python
API. I also used the Python API to create a depletion chain from the
spontaneous and delayed fission yield data, decay data, and neutron cross
secttion data from the ENDF B/VII.1 library. Interested readers are able to
create this library files by running the scripts located at
\url{https://github.com/arfc/saltproc/tree/master/scripts/xsdata}.\footnote{See
the \verb,README.md, in the parent directory for user instructions}.

We assumed material temperatures to be around 900K. For cross section data
unavailable at that temperature, we used interpolation between 800K and 1000
to get reasonable values.
 

\section{Comparison of OpenMC to Serpent}
\label{sec:openmc-vs-serpent}

