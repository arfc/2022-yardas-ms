\chapter{Introduction}%
\label{cha:introduction}

\section{Motivation}%
\label{sec:motivation}

Increasing \Gls{ghg} emissions due to human activity since the Industrial Revolution drives observed increases in average global surface temperature via the greenouse effect\cite{mitchell_greenhouse_1989} \cite{paola_a_arias_2021_ts} (\textbf{global warming}).
The long term impacts of global warming...
Rapid decrease in global \Gls{ghg} emissions can dampen the impact of global warming. Transitioning from fossil-fuel-based electrictity production methods to zero-emission \footnote{zero-emissions during operation; there are lifecycle carbon costs associated with any form of power production} electrcity production methods (\textbf{decarbonization}) would contribute significantly to a global \Gls{ghg} emission reduction, and in turn dampen the effects of global
warming. Our three main options for decarbonization are renewables (solar, wind, geothermal), hydropower, and nuclear power. We will need all of these technologies working together if we want to sucessfully decarboinze, however each technology has a specifc use case where it works best; solar and wind are useful in responding to rapid fluctuations in electicity usage, wheras nuclear, hydropower, and geothermal are better suited to
providing a baseload level of electricity (cite). Geothermal and hydropower are limited by geographical features (geothermal acticivity and elevated water sources), making nuclear the most flexibile option for a future carbon-free baseload. Therefore, we should maintian and develop nuclear power technologies to fight against global warming.

\section{Current and future trajectories of nuclear power}%
\label{sec:current_and_future_trajectories_of_nuclear_power}
Nuclear power currently constitutes a fifth of US domestic electricty production, and half of US zero-carbon electricity production (cite EIA). Unfortunatley, our current nuclear fleet faces several threats to its long-term survival. The most relevant of these is the state of the electricity market: record-low natural gas prices, increases in subsidies for renewables, and general lack of similar compensation for
nuclear power plants make operating nuclear power plants financially unsustainable in the long term given the current market conditions\footnote{This is a generalization as this issue is incredibly complex and requires more space that I can give to it in this thesis to fully appreciate} \cite{szilard_economic_2016}. %perhaps mention all plants that have been shut down recently?

Assuming that we resolve the economic threats to our current nuclear fleet, they are -- like all things -- subject to aging and deterioration; at some point in the future, we will need to shut down and decommission them. As explained in the previous section, nuclear power will be a key player in any decarbonization strategy. We will need to replace the old plants with new ones that are more sustainable, economically competitive, safe, and reliable than
their predecessors.
%% I want to make this point differently... the conclusion should be that we need new nuclear power tech to ensure we can decarbonize, but I don't feel like the preceding statements smoothly come to that conclusion in the way I have preseneted them here
 
\section{Generation IV Reactors}%
\label{sec:generation_iv_reactors}

The first generation of nuclear power reactors includes the first prototpyes and early civil deployments in pursuit of cheap and bountiful energy. The second generation of reactors were built on top of this momentum, as well as (in some cases) in response to the 1973 oil crisis (cite). Following the well publicized and documented accidents at Three Mile Island (1976) and Chernobyl (1981) power plants, the third generation of nuclear power was developed with increased saftey and reactor lifetime
in mind. The fourth generation of nuclear power
reactors will need to rapidly respond to increasing electricity consumption and the need for decarbonzation. This means the fourth generation of nuclear reactors need to be able to be built quickly, have widespread use, all while maintaining or increasing fuel efficiency, economic competitivness, and saftey and proliferation resistance.

This is essentially the conclusion that the \Gls{gif} -- a "co-operative international endeavour seeking to develop the feasability and performance of routh generation nuclear systems" \cite{gif_homepage} -- came to in their 2002 roadmap \cite roadmap. This efford selected six reactor technologies in total. One of these technologies, the \Gls{msr}, is the central technology of this thesis.
%% why the msr??

The technological complexities of the Molte


- several things threatening current fleet
  - age
  - low ng prices
  - public opinion
  - loss of building experience

- need for 
- current nuclear fleet is very old, nat gas prices
 -> assert: nuclear power can do one of two things
    1. Current tech -> replace large nat-gas power stations 
    2. "new" nuclear -> relpace smaller power stations + expand scope
        of nuclear power to more specialized use cases (space, remote communities, etc)
-> six gen4 reactors -> molten salt reactor one of these
-> NRC liscening approach for advacned reactors will be based on current R\&D, specifcially M\&S
-> M\&S tools that answer all relevant licesning questions for msr specifically do not exist
-> SaltProc aims to be a tool that can answer some of these questions, as well as a general purpose
    research tool for simulating MSRs

\section{Objectives}%
\label{sec:objectives}

 free and open source software provides lowest barrier of entry
 - saltproc built for use with serpent which defeats the purpose of being
    FOSS
 - Code needs to be restructured to enable easier development of interfaces to 
 open source MC transport codes
 - We then need to implement the necessary machinery to interface with a specific
    open source MC transport code (OpenMC)
 - Once implemented, we need to validate with a serpent-based case
