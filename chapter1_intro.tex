\chapter{Introduction}%
\label{cha:introduction}

\section{Motivation}%
\label{sec:motivation}

Increasing \Gls{ghg} emissions due to human activity since the Industrial Revolution drives observed increases in average global surface temperature via the greenouse effect\cite{mitchell_greenhouse_1989} \cite{paola_a_arias_2021_ts} (\textbf{global warming}).
The long term impacts of global warming...
Rapid decrease in global \Gls{ghg} emissions can dampen the impact of global warming. Transitioning from fossil-fuel-based electrictity production methods to zero-emission \footnote{zero-emissions during operation; there are lifecycle carbon costs associated with any form of power production} electrcity production methods (\textbf{decarbonization}) would contribute significantly to a global \Gls{ghg} emission reduction, and in turn dampen the effects of global
warming. Our three main options for decarbonization are renewables (solar, wind, geothermal), hydropower, and nuclear power. We will need all of these technologies working together if we want to sucessfully decarboinze, however each technology has a specifc use case where it works best; solar and wind are useful in responding to rapid fluctuations in electicity usage, wheras nuclear, hydropower, and geothermal are better suited to
providing a baseload level of electricity (cite). Geothermal and hydropower are limited by geographical features (geothermal acticivity and elevated water sources), making nuclear the most flexibile option for a future carbon-free baseload. Unfortunatley, nuclear power is a divisive technology(cite), with both legitimate technological and policy concerns.%consider expanding on this?
Luckily, many of these concerns have technological solutions. Therefore, we should maintian and develop nuclear power technologies so that they can contribute to decarbonization efforts.

\section{Current and future trajectories of nuclear power}%
\label{sec:current_and_future_trajectories_of_nuclear_power}
Nuclear power currently constitutes a fifth of US domestic electricty production, and half of US zero-carbon electricity production (cite EIA). Unfortunatley, our current nuclear fleet faces several threats to its long-term survival. The most relevant of these is the state of the electricity market: record-low natural gas prices, increases in subsidies for renewables, and general lack of similar compensation for
nuclear power plants make operating nuclear power plants financially unsustainable in the long term given the current market conditions\footnote{This is a generalization as this issue is incredibly complex and requires more space that I can give to it in this thesis to fully appreciate} \cite{szilard_economic_2016}. %perhaps mention all plants that have been shut down recently?

Assuming that we resolve the economic threats to our current nuclear fleet, they are -- like all things -- subject to aging and deterioration; at some point in the future, we will need to shut down and decommission them. As explained in the previous section, nuclear power will be a key player in any decarbonization strategy even though it is controversial. To both address this controversy and maintain nuclear power's postion in our decarbonization technology stack, we will need to develop and
implement new kinds of nuclear reactors that are more sustainable, economically competitive, safe, and reliable than
their predecessors.
%% I want to make this point differently... the conclusion should be that we need new nuclear power tech to ensure we can decarbonize, but I don't feel like the preceding statements smoothly come to that conclusion in the way I have preseneted them here
 
\section{Molten Salt Reactors}%
\label{sec:molten_salt_reactors}

The first generation of nuclear power reactors includes the first prototpyes and early civil deployments in pursuit of cheap and bountiful energy. The second generation of reactors were built on top of this momentum, as well as (in some cases) in response to the 1973 oil crisis (cite). Following the well publicized and documented accidents at Three Mile Island (1976) and Chernobyl (1981) power plants, the third generation of nuclear power was developed with increased saftey and reactor lifetime
in mind. The fourth generation of nuclear power
reactors will need to rapidly respond to increasing electricity consumption and the need for decarbonzation. This means the fourth generation of nuclear reactors need to be able to be built quickly, have widespread use, all while maintaining or increasing fuel efficiency, economic competitivness, and saftey and proliferation resistance.

This is essentially the conclusion that the \Gls{gif} -- a "co-operative international endeavour seeking to develop the feasability and performance of routh generation nuclear systems" \cite{gif_homepage} -- came to in their 2002 roadmap \cite{doe-ne_technology_2002}. This effort selected six reactor technologies in total. One of these technologies, the \Gls{msr}\footnote{for this thesis {\it Molten Salt Reactor} refers specifically to the reactor type where the fissile material is dissolved in in the salt
coolant}, is of particular interest due to the unique challenges and opportunities is presents.
%% why the msr??

The \Gls{msr} is so named as it is a nuclear reactor that uses a mixture of liquid-phase salts as a coolant. The use of a liquid fuel enables adoption of {\it on-line reprocessing} by pumping used fuel out of the reactor and pumping fresh or reprocessed fuel back into the reactor\footnote{this concept is shown as {\it separations and feeds}}. This enables removal of undesirable neutron absorbers created in the fission reacton from the fissile material. In contrast, while solid-fueled reactors will shuffle
    the fuel within the assembly, fission products remain trapped in the fuel, and can reduce the thermal performance of the fuel via thermal cracking as well as absorb neutrons that would otherwise contribute to the fission reaction. 

    \Gls{msr}s face several techical and logistical challenges before they could be deployed for civillian power generation. High temperature liquid-phase salt will steadily corrode metals over time, so  the reactor vessel of a \Gls{msr} must use special corrosion-resistant materials. High temperature salt itself is extremley hazardous and reacts explosively with moisture, so special procedures and PPE must be used when handling fuel salt, or fuel salt must be handled entirely
    remoteley. Even with these challenges, I believe the potential benefits of \Gls{msr} technology merit its development.

\Gls{mns} codes will play a critical role in lisencing GenIV reactors. In preparation for \Gls{msr}s, both the \Gls{doene} and the \Gls{nrc} have identified several technical gaps in current \Gls{mns} tools that are necessary for efficient and effective license application reviews \cite{betzler_modeling_2019} \cite{usnrc_nonlwr_2020-1}. In particular, both the \Gls{doe} and the \Gls{nrc} have identified the fuel composistion and its evolution in a \Gls{msr} to be a key code feature
necessary for accident analysis. 

\section{MSR Depletion Codes}%
\label{sec:msr_codes}

To model the changing fuel composition of the fuel in a \Gls{msr}, there are at least two processes we must consider:
\begin{enumerate}
    \item Fuel {\it depletion}
    \item Removal and feed processes
\end{enumerate}

more realistic simulations may incorporate computational fluid dynamics to model the dynamical behavior of the fluid itself.

Several efforts modeling fuel composition evolution in \Gls{msr}s currently exist (see Table \ref{tab:msr_codes})

of \Gls{msr}s:
%you'll need to read these papers to refresh your memory a bit
\begin{itemize}
    \item ChemTriton (cite chem triton paper)
    \item Serpent2 (cite auferio)
    \item SaltProc (cite Rykhlevskii)
\end{itemize}

Among these tools, only one of them (SaltProc) is open source. Open source nuclear software is\ldots (benfits of open source) 
Now, while SaltProc itself is open source, previously relied exclusivley on Serpent2 to perform depletion calculations.

%
%
%

There are several software tools in active development that have the capability to simulate the fuel composition and evolutions in an \Gls{msr}.
% list in the table below?

%andrei talks about these codes in his thesis. Do I need to repeat that here? Obviously I need to put my own spin on it...

SaltProc is similar to ChemTriton. SaltProc is unique among its peers as it has an open source codebase and development process. This is significant as open source software lowers the barriers to distribution, development, and usage that \Gls{cc}s present \cite{fiorina_initiative_2021}. Historically, software used in licensing, \Gls{rnd}, and \Gls{ent} efforts in the nuclear field have been closed source and proprietary (cite). The number of open source projects in this industry (cite a bunch of open source nuclear software:
ONIX, OpenMC, NJOY21, cyclus, \ldots) is growing in recognition of the need for distributable, high-quality, and transparent software tools. This is perhaps best seen in a \Gls{iaea} facilitated \Gls{oncore} initiative to "[promote] development and application of open-source multi-physics simulation tools to support research, education, and training for analysis of advanced nuclear power reactors" \cite{iaea_open-source}. It is an exciting time to be a software developer in the nuclear field!

\section{Objectives}%
\label{sec:objectives}

While SaltProc itself is open source, Serpent is not. SaltProc being open source does lower the barrier of entry for simulating depletion in a \Gls{msr}, however this point is almost moot when the main dependency to do depletion calulations is not. It is in this spirit that I have added support for OpenMC \cite{romano_openmc_2015}, an open source monte-carlo particle transport code.

This Master's thesis has two primary objectvies
\begin{enumerate}
    \item Refactor SaltProc for to enable OpenMC support, as well as enable easier implementation of support for other monte carlo codes in the future. 
    \item Verify the implementation on a well studied MSR system.
\end{enumerate}


- several things threatening current fleet
  - age
  - low ng prices
  - public opinion
  - loss of building experience

- need for 
- current nuclear fleet is very old, nat gas prices
 -> assert: nuclear power can do one of two things
    1. Current tech -> replace large nat-gas power stations 
    2. "new" nuclear -> relpace smaller power stations + expand scope
        of nuclear power to more specialized use cases (space, remote communities, etc)
-> six gen4 reactors -> molten salt reactor one of these
-> NRC liscening approach for advacned reactors will be based on current R\&D, specifcially M\&S
-> M\&S tools that answer all relevant licesning questions for msr specifically do not exist
-> SaltProc aims to be a tool that can answer some of these questions, as well as a general purpose
    research tool for simulating MSRs


 free and open source software provides lowest barrier of entry
 - saltproc built for use with serpent which defeats the purpose of being
    FOSS
 - Code needs to be restructured to enable easier development of interfaces to 
 open source MC transport codes
 - We then need to implement the necessary machinery to interface with a specific
    open source MC transport code (OpenMC)
 - Once implemented, we need to validate with a serpent-based case
