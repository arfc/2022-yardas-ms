\chapter{Introduction}%
\label{cha:introduction}

\section{Motivation}%
\label{sec:motivation}

Increasing \Gls{ghg} emissions due to human activity since the Industrial Revolution drives observed increases in average global surface temperature via the greenouse effect\cite{mitchell_greenhouse_1989} \cite{paola_a_arias_2021_ts} (\textbf{global warming}).
The long term impacts of global warming...
Rapid decrease in global \Gls{ghg} emissions can dampen the impact of global warming. Transitioning from fossil-fuel-based electrictity production methods to zero-emission \footnote{zero-emissions during operation; there are lifecycle carbon costs associated with any form of power production} electrcity production methods (\textbf{decarbonization}) would contribute significantly to a global \Gls{ghg} emission reduction, and in turn dampen the effects of global
warming. Our three main options for decarbonization are renewables (solar, wind, geothermal), hydropower, and nuclear power. We will need all of these technologies working together if we want to sucessfully decarboinze, however each technology has a specifc use case where it works best; solar and wind are useful in responding to rapid fluctuations in electicity usage, wheras nuclear, hydropower, and geothermal are better suited to
providing a baseload level of electricity (cite). Geothermal and hydropower are limited by geographical features (geothermal activity and elevated water sources), making nuclear the most flexibile option for a future carbon-free baseload. Despite these positive qualities, nuclear power is a divisive technology(cite), with both legitimate technological and policy concerns.%consider expanding on this?
Luckily, many of these concerns have technological solutions. Therefore, we should maintian and develop nuclear power technologies so that they can contribute to decarbonization efforts.

\section{Current and future trajectories of nuclear power}%
\label{sec:current_and_future_trajectories_of_nuclear_power}
Nuclear power currently constitutes a fifth of US domestic electricty production, and half of US zero-carbon electricity production (cite EIA). Unfortunatley, our current nuclear fleet faces several threats to its long-term survival. The most relevant of these is the state of the electricity market: record-low natural gas prices, increases in subsidies for renewables, and general lack of similar compensation for
nuclear power plants make operating nuclear power plants financially unsustainable in the long term given the current market conditions\footnote{This is a generalization as this issue is incredibly complex and requires more space that I can give to it in this thesis to fully appreciate} \cite{szilard_economic_2016}. %perhaps mention all plants that have been shut down recently?

Assuming that we resolve the economic threats to our current nuclear fleet, they are -- like all things -- subject to aging and deterioration; at some point in the future, we will need to shut down and decommission them. As explained in the previous section, nuclear power will be a key player in any decarbonization strategy even though it is controversial. To both address this controversy and maintain nuclear power's postion in our decarbonization technology stack, we will need to develop and
implement new kinds of nuclear reactors that are more sustainable, economically competitive, safe, and reliable than
their predecessors.
%% I want to make this point differently... the conclusion should be that we need new nuclear power tech to ensure we can decarbonize, but I don't feel like the preceding statements smoothly come to that conclusion in the way I have preseneted them here
 
\section{Molten Salt Reactors}%
\label{sec:molten_salt_reactors}

The first generation of nuclear power reactors includes the first prototpyes and early civil deployments in pursuit of cheap and bountiful energy. The second generation of reactors were built on top of this momentum, as well as (in some cases) in response to the 1973 oil crisis (cite). Following the well publicized and documented accidents at Three Mile Island (1976) and Chernobyl (1981) power plants, the third generation of nuclear power was developed with increased saftey and reactor lifetime
in mind. The fourth generation of nuclear power
reactors will need to rapidly respond to increasing electricity consumption and the need for decarbonzation. This means the fourth generation of nuclear reactors need to be able to be built quickly, have widespread use, all while maintaining or increasing fuel efficiency, economic competitivness, and saftey and proliferation resistance.

This is essentially the conclusion that the \Gls{gif} -- a "co-operative international endeavour seeking to develop the feasability and performance of routh generation nuclear systems" \cite{gif_homepage} -- came to in their 2002 roadmap \cite{doe-ne_technology_2002}. This effort selected six reactor technologies in total. One of these technologies, the \Gls{msr}\footnote{for this thesis {\it Molten Salt Reactor} refers specifically to the reactor type where the fissile material is dissolved in in the salt
coolant}, is of particular interest due to the unique challenges and opportunities is presents.
%% why the msr??

The \Gls{msr} is so named as it is a nuclear reactor that uses a mixture of liquid-phase salts as a coolant. The use of a liquid fuel enables adoption of {\it on-line reprocessing} by pumping used fuel out of the reactor and pumping fresh or reprocessed fuel back into the reactor\footnote{this concept is shown as {\it separations and feeds}}. This enables removal of undesirable neutron absorbers created in the fission reacton from the fissile material. In contrast, while solid-fueled reactors will shuffle
    the fuel within the assembly, fission products remain trapped in the fuel which can reduce the thermal performance of the fuel via thermal cracking as well as absorb neutrons that would otherwise contribute to the fission reaction. 

    \Gls{msr}s face several techical and logistical challenges before they could be deployed for civillian power generation. High temperature liquid-phase salt will steadily corrode metals over time, so  the reactor vessel of a \Gls{msr} must use special corrosion-resistant materials. High temperature salt itself is extremley hazardous and reacts explosively with moisture, so special procedures and PPE must be used when handling fuel salt, or fuel salt must be handled entirely
    remoteley. Even with these challenges, I believe the potential benefits of \Gls{msr} technology merit its development.

\Gls{mns} codes will play a critical role in lisencing GenIV reactors. In preparation for \Gls{msr}s, both the \Gls{doene} and the \Gls{nrc} have identified several technical gaps in current \Gls{mns} tools that are necessary for efficient and effective license application reviews \cite{betzler_modeling_2019} \cite{usnrc_nonlwr_2020-1}. In particular, both the \Gls{doe} and the \Gls{nrc} have identified the fuel composistion and its evolution in a \Gls{msr} to be a key code feature
necessary for accident analysis. 

\section{MSR Depletion Codes}%
\label{sec:msr_codes}

To model the changing fuel composition of the fuel in a \Gls{msr}, there are at least two processes we must consider:
\begin{enumerate}
    \item Fuel {\it depletion}\footnote{the consumption of fissile material in the fuel and production of fission products via the fission chain reaction}
    \item Removal and feed processes
\end{enumerate}

The Bateman equation describes the rate of change of the number density of a nuclide in a nuclear reaction mathematically:

\begin{align}
    \label{eq:bateman-1}
    \frac{dN_{i}}{dt} =& \sum_{j} l_{j\to i}\lambda_{j} N_{j} + \gamma_{i} \Sigma^{f}\phi + \phi N_{i-1} \sigma^{a}_{i-1} - \lambda_{i}N_{i} - N_{i}\sigma{a}_{i}\phi\\
    N_{\*} =& \text{number density for nuclide $\*$ $[cm^{-3}]$}\nonumber\\
    l_{j\to i} =& \text{branching ratio for decay mode of nuclide $j$ that produces nuclide $i$}\nonumber\\
    \lambda_{\*} =& \text{decay constant of nuclide $\*$ $[s^{-1}]$}\nonumber\\
    \gamma_{i} =& \text{fission yield fraction for nuclide $i$}\nonumber \\
    \Sigma^{f} =& \text{average macroscopic fission cross section $[cm^{-1}]$}\nonumber\\
    \phi =& \text{neutron flux $[cm^{-2}s^{-1}]$}\nonumber\\
    \sigma^{a}_{i-1} =& \text{neutron absorption cross-section for nuclide $i-1$ $[cm^{-2}]$}\nonumber\\
\end{align}
    
This equation sufficiently describes the process of depletion for any nuclide $i$. When solving for $n$ nuclides, we solve the matrix problem $\frac{d}{dt}N = \mathbf{A}N$, where $N$ is a $n$-vector and $\mathbf{A}$ contains all the coefficient terms in equation \ref{eq:bateman-1}. Incorporating removals and feeds into this equation involves the addition of a time dependent removal factor $r_{i}(t)$ and feed factor $f_{i}(t)$ to equation \ref{eq:bateman-1}. The resulting equation describes {\it continuous reprocessing}.
For $n$ nuclides, the matrix problem is then $\frac{d}{dt}N = \mathbf{A}N + S(t)$, where $S$ is a $n$-vector containing the sums of the removal and feed terms for each nuclide $i$. The additonal term in the Bateman equation from continuous reprocessing increases computational cost and implementation difficulty, and may require a completley different set of precondtioners for numberical stability and convergence.

Alternatively, one could run a depletion simulation, perform the removals and feeds in an external appplication on the resulting material composition,and run another depletion simulation on the reprocessed composition. This procedure models {\it batchwise reprocessing}, where material from the core is reprocessed at regular intervals rather than continuously. This requires coupling to an external piece of software, which comes with its own challenges, but benefit of this approach is that the external softare could
support {\it any} software capable of doing depletion calculations.

SaltProc\cite{rykhlevskii_saltproc_2018}, the focus of the current work, uses a batchwise reprocessing approach to model the fuel composition in an \Gls{msr} and uses Serpent2\cite{leppanen_serpent_2014} to run depletion simulations. SaltProc is unique among its peers as it has an open source codebase and development process \footnote{There are other software projects that model fuel composition in \Gls{msr}s. Notably, ChemTriton\cite{betzler_molten_2017} -- a python script for SCALE/TRITON -- is
functionally similar to SaltProc. Section 1.2 in \cite{rykhlevskii_fuel_2020} and section 4.2 in \cite{rykhlevskii_advanced_2018} provide a high-level summary of other recent efforts}. It is my hope that SaltProc can be used both as a general purpose research tool as well as answer some of the important questions regarding fuel composition evolution in an \Gls{msr} for liscening purposes. 

Historically, software used in licensing, \Gls{rnd}, and \Gls{ent} efforts in the nuclear field has been closed source and proprietary. For \Gls{rnd} and \Gls{ent} efforts in particular, this can bring collaborative efforts to a grinding halt until regulatory bodies grant software licences. Using \Gls{cc}s in scientific publications and research presents barriers to reproducibility and the ability of external verification of results.
    
    Regulatory bodies will require new software features (and in some cases entirely new software tools) in order to effectively and efficiently perform licensing activities for the next generation of advanced reactor designs \cite{usnrc_nonlwr_2020-1}, and many of the open source tools emerging in the past decade (e.g. OpenMC\cite{romano_openmc_2015}) have the advantage over their legacy closed code ancestors (e.g. Serpent \cite{leppanen_serpent_2014}) of using best-practices for software development. It follows that these features and tools are more readily implementable in these new open source projects than in the legacy closed codes.
    
    We are entering the era of \Gls{oss} purpose-built for applications in nuclear science and engineering. The number of open source projects in this industry (ONIX\cite{de_troullioud_de_lanversin_onix_2021}, OpenMC, NJOY21\cite{noauthor_njoy21_2022}, Cyclus\cite{noauthor_cyclus_2022}, to name a few\footnote{the awesome-nuclear repository on GitHub \cite{romano_awesome_2022} has good list of nuclear-related open source software projects.}) is growing in recognition of the need for distributable, high-quality, and transparent software tools. This is perhaps best seen in the \Gls{iaea} facilitated \Gls{oncore} initiative \cite{fiorina_initiative_2021} to "[promote] development and application of open- source multi-physics simulation tools to support research, education, and training for analysis of advanced nuclear power reactors"  \cite{iaea_open-source_2022}. 


\section{Objectives}%
\label{sec:objectives}

While SaltProc itself is open source, Serpent2 is not. OpenMC recently added a \verb.deplete. module for fuel depletion simulations, meanding it is now possible to have a fully open-source stack of dependencies for SaltProc.  It is in this spirit that I have added support for OpenMC to SaltProc. This improves the accessibility and usability of SaltProc, and I hope that researchers in this field will begin using and contributing to the tool.

This Master's thesis has two primary objectvies
\begin{enumerate}
    \item Refactor SaltProc for to enable OpenMC support, as well as enable easier implementation for other monte carlo depletion codes in the future. 
    \item Verify the implementation on a well studied MSR system.
\end{enumerate}
