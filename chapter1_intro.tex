\chapter{Introduction}%
\label{cha:introduction}

\section{Motivation}%
\label{sec:motivation}

Increasing \Gls{ghg} emissions due to human activity since the Industrial Revolution drive observed increases in average global surface temperature via the greenouse effect\cite{mitchell_greenhouse_1989} \cite{paola_a_arias_2021_ts}(global warming).
The long term impacts of global warming...
Rapid decrease in global \Gls{ghg} emissions can dampen the impact of global warming. Decarbonization of our electrictiy production methods would contribute significantly to global \Gls{ghg} emissions. Our three main options for decarbonization are renewables (solar, wind, geothermal), hydropower, and nuclear power. Whereas solar and wind aid in responding to rapid fluctuations in electicity usage, nuclear, hydropower, and geothermal are better suited to
providing a baseload level of electricity. Geothermal and hydropower are limited by geographical features (geothermal acticivity and elevated water sources), making nuclear the most flexibile option for a future carbon-free baseload.

\section{Current and future trajectories of nuclear power}%
\label{sec:current_and_future_trajectories_of_nuclear_power}
Nuclear power currently constitutes a fifth of US domestic electricty production, and half of US zero-carbon\footnote{zero-emissions during operation; there are lifecycle carbon costs associated with any form of power production} electricity production. Our current nuclear fleet is aging, and will likeley need to be decommissioned within the next 20-40 years.


 -> assert: nuclear power can do one of two things
    1. Current tech -> replace large nat-gas power stations 
    2. "new" nuclear -> relpace smaller power stations + expand scope
        of nuclear power to more specialized use cases (space, remote communities, etc)
-> six gen4 reactors -> molten salt reactor one of these
-> NRC liscening approach for advacned reactors will be based on current R\&D, specifcially M\&S
-> M\&S tools that answer all relevant licesning questions for msr specifically do not exist
-> SaltProc aims to be a tool that can answer some of these questions, as well as a general purpose
    research tool for simulating MSRs

\section{Objectives}%
\label{sec:objectives}

 free and open source software provides lowest barrier of entry
 - saltproc built for use with serpent which defeats the purpose of being
    FOSS
 - Code needs to be restructured to enable easier development of interfaces to 
 open source MC transport codes
 - We then need to implement the necessary machinery to interface with a specific
    open source MC transport code (OpenMC)
 - Once implemented, we need to validate with a serpent-based case
