\chapter{Software Overview and Development}%
\label{cha:software_development}

A major component of this work was restructuring of the SaltProc code and implementation of OpenMC.
\section{SaltProc}%
\label{sec:saltproc}

SaltProc\cite{rykhlevskii_saltproc_2018} is an open source Python package that simulates on-line reprocessing via a batch-wise approach\footnote{Material is moved to or from the core at specific time intervals} in liquid-fueled \Gls{msr}s. More precisely, SaltProc manages material flows and separation processes on nuclides in the fuel. SaltProc relies on external codes to simulate fuel depletion.

The first version of SaltProc (v0.1) was a simple Python 2.7 package that used SERPENT 2 for the fuel depletion simulations. A single Python file contained all functions; separation processes applie dto  used an implicit 100\% efficiency. The structure of SaltProc v0.1...   

\section{OpenMC}%
\label{sec:openmc}

OpenMC \cite{romano_openmc_2015} is an open source Monte Carlo particle transport code. The \Gls{crpg} at \Gls{mit} started developing OpenMC back in 2011 with a focus on scalability for exascale computing. Since that time, developers new and old contributed features (cite?) and fixes to the tool expanding its scope and use cases. Notable features of OpenMC (as of version 0.12.1) are as follows \cite{openmc_homepage}:
\begin{itemize}
    \item Support for fixed source, $k$-eigenvalue, and subcritical neutron multiplication cacluclations.
    \item Support for \Gls{csg} and \Gls{cad} geometry.
    \item Support for both continuous and multigroup transport calculations.
    \item Support for parallel execution via MPI and OpenMP.
    \item Geometry visualization through the Python API.
\end{itemize}
The tool is now quite mature and feature-rich, rivaling its closed-source export controlled counterparts. 


Recently, depletion and photon transport were added by (who?)... The new depletion feature enables us to couple OpenMC to SaltProc and a fully open-source stack.
