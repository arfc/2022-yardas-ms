\chapter{Model Description}
\label{ch:chapter4}
As mentioned in Chapter 1, I am using a \Gls{csg} model of the \Gls{msbr}
\cite{robertson_conceptual_1971} to verify my \OpenMC implementation in
\SaltProc.

I picked the \Gls{msbr} because\ldots 

The \Gls{msbr} design is the result of a design study of a single-fluid
\Gls{msr} following the success of the \Gls{msre}
\cite{haubenreich_experience_1970}\cite{rosenthal_molten-salt_1970}.
I will only describe the following reactor systems that are relevant to
my validation study\footnote{A complete description of the entire \Gls{msbr}
system can be found in \cite{robertson_conceptual_1971}}: the fuel salt, the
reactor core, and the salt reprocessing system.

\begin{figure}[htpb] 
    \centering
    \subfloat[][]{
        \includegraphics[width=0.5\linewidth]{figs/ch4/msbr_full_xy_ref.png}
        \label{fig:msbr_ref_xy}
    }
    \subfloat[][]{
        \includegraphics[width=0.5\linewidth]{figs/ch4/msbr_full_xy_openmc.png}
        \label{fig:msbr_model_xy}
    }
    \\
    \subfloat[][]{
        \includegraphics[width=0.5\linewidth]{figs/ch4/msbr_full_xz_ref.png}
        \label{fig:msbr_ref_xz}
    }
    \subfloat[][]{
        \includegraphics[width=0.5\linewidth]{figs/ch4/msbr_full_xz_openmc.png}
        \label{fig:msbr_model_xz}
    }
    \caption[Full views of MSBR]{
        \subref{fig:msbr_ref_xy} Top down view of \Gls{msbr} reference design.
        \subref{fig:msbr_model_xy} Top down view of \Gls{msbr} CSG model.
        \subref{fig:msbr_ref_xz} Top down view of \Gls{msbr} reference design.
        \subref{fig:msbr_model_xz} Top down view of \Gls{msbr} CSG model.
    }
    \label{fig:msbr-overview}
\end{figure}

As seen in Figure \ref{fig:msbr-overview} \OpenMC and \SerpentTWO \Gls{msbr}
models of reproduce these systems with several approixmations. I will
describe each reactor system, as well as any relavant changes or
approximations made in the model.

\section{Materials}
\label{sec:msbr-materials}

\subsection{Fuel salt}
\label{sub:msbr-fuel-salt}
Table S.1 in \cite{robertson_conceptual_1971} specifies the fuel salt
composition used in the \Gls{msbr}:
\ce{LiF}-\ce{Be}\ce{F_2}-\ce{Th}\ce{F_4}-\ce{U}\ce{F_4} at a
concentration of 71.7-16-12-0.3 mole-\%\footnote{In Rykhlevskii's thesis
\cite{rykhlevskii_fuel_2020}, he stated the mole-\% to be 71.75-16-12-0.25. I
have been unable to find this composition in Robertson et al.
\cite{robertson_conceptual_1971}}. The lithium used in the fuel salt is
enriched to 99.995\% \ce{^{7}Li}. This is because \ce{^{6}Li} is a strong
neutron absorber and produces tritium in the absorption reaction. The atom-\%
for each nuclide is given in Table \ref{tab:msbr_fuel_salt-ref}\footnote{Most of the
discussion of the fuel salt compositon in Robertson el al
\cite{robertson_conceptual_1971} specify elemental version of the nuclides in
Table \ref{tab:msbr_fuel_salt-ref}. I have specific specific nuclides for the
following reasons: (1) both \ce{F} and \ce{Be} have only one stable isotope, (2)
the fuel salt recieves initial fissile loading from \ce{^{233}U} or
\ce{^{235}U}, and (3) the fuel salt recieves its fertile loading from
\ce{^{232}Th}}.

\begin{table}[htpb] 
    \centering 
    \caption{Reference \Gls{msbr} fuel salt specifications}
    \label{tab:msbr_fuel_salt-ref}
    \begin{tabular}{|c|c|c|c|c|c|c|} 
        \hline
        & \ce{^{6}Li} & \ce{^{7}Li} & \ce{^{19}F} & \ce{^{9}Be} & \ce{^{232}Th} & \ce{^{233}U}\\
        \hline 
        atom-\% & 1.4357925 & 34.4142075 & 56.35$\overline{6}$ & 5.$\overline{3}$ & 2.4 & 0.06 \\
        \hline
        mass-\% & 0.4452449 & 12.4477343 & 55.1982285 & 2.4779439 & 28.7100003 & 0.7208481\\ 
        \hline
    \end{tabular}
\end{table}

The density of the fuel salt is given by a function\footnote{This function 
comes from an earlier report on the Molten-Salt Reactor Program
\cite{rosenthal_molten-salt-ornl_1970}.} of temperature in \unit{\celsius} in Table
S.1 in Robertson et al. \cite{robertson_conceptual_1971}:
\begin{equation}
    \rho = 3.752 - 6.68\cdot 10^{-4} \cdot T \quad \unit{\gram\per\square  \centi\meter}
\end{equation}

The temperature of the fuel salt flowing into the core at the inlet at the
bottom of the reactor is 1050\unit{\degree}F (565.5556\unit{\celsius}, 838.7056
\unit{\kelvin}), and the temperature of the fuel salt flowing out of the core at
the outlet is approximately 1300\unit{\degree}F (704.4444\unit{\celsius},
977.5944 \unit{\kelvin})\cite{robertson_conceptual_1971}. The average
temperature of the salt over the core inlets and outlets is then 1175
\unit{\degree}F (635\unit{\celsius}, 908.15 \unit{\kelvin}). While the
various solid components of the core are at a slightly higer temperature on
average\footnote{see figure 3.29 in \cite{robertson_conceptual_1971}}, for
simplicity, I set the evaluated temperature of all materials to 900
\unit{\kelvin} consistent cross-section selection between the \OpenMC and
\SerpentTWO depletion steps. At this temperature, the density of the fuel salt
is 3.3332642 \unit{\gram\per\centi\metre\cubed}.

\begin{table}[htpb] 
    \centering 
    \caption{Model \Gls{msbr} fuel salt specifications}
    \label{tab:msbr_fuel_salt-model}
    \begin{tabular}{|c|c|c|c|c|c|} 
        \hline
        & \ce{^{7}Li} & \ce{^{19}F} & \ce{^{9}Be} & \ce{^{232}Th} & \ce{^{233}U}\\
        \hline 
        atom-\% & 28.386326 & 60.437153 & 6.330058 & 4.747557 & 0.098907 \\
        \hline
        mass-\% & 7.87474673879085 & 45.4003012179284 & 2.25566879138321 & 43.5579130482336 & 0.911370203663893\\ 
        \hline
    \end{tabular}
\end{table}
In the CSG model, the fuel salt material uses the composition specified in
Table \ref{tab:msbr_fuel_salt-model} and has a density of 3.35
\unit{\gram\per\centi\metre\cubed}. Notice that the lithium has been enriched to
100\% \ce{^{7}Li}. This is because during inital simulations, even those very
small amounts of \ce{^{6}Li} were enough to kill the reaction. The inital
fissile and fertile loading has also been slightly increased.

\subsection{Graphite}
\label{sub:graphite}

For a detailed description of the reactor graphite used in the \Gls{msbr}, see
Section 3.2.3 in \cite{robertson_conceptual_1971}. At 70\unit{\degree}F (294.3
\unit{\kelvin}), the \Gls{msbr} graphite has a density of 1843
\unit{\kilo\gram\per\cubic\metre}. This is the only density specification
for graphite that I was able to find in Robertson et al.
\cite{robertson_conceptual_1971}

In the CSG model, the graphite material uses elemental carbon and has a
density of 1.84\unit{\gram\per\centi\metre\cubed}.

\subsection{Modified Hastelloy N}
\label{sub:hastelloy}
Hastelloy N is an alloy developed at \Gls{ornl} during the Molten-Salt Reactor
Program as a structural material that could maintain structural stability while
in contact with the corrosive and high temperature molten salt fuel while also
being under irradation for a long period of time.

The \Gls{msbr} used a modified version of Hastelloy N designed to improve
embrittlement resistance and weldibility \cite{robertson_conceptual_1971}.
The \Gls{msbr} uses modified Hastelloy N on all nearly all salt-facing
components included in the CSG model.

Modified Hastelloy N has a density of 8671 \unit{\kilo\gram\per\cubic\metre} at
1300\unit{\degree}F (704.4444\unit{\celsius}, 977.5944 \unit{\kelvin})
\cite{robertson_conceptual_1971}. The elemental composition of modified
Hastelloy N and their amounts in mass-\% are in Table \ref{tab:hastelloy-n-ref}.

\begin{table}[htpb]
    \centering
    \caption[Mass-\% of elements in modified Hastelloy N used in the \Gls{msbr}]{Mass-\% of elements in modified Hastelloy N used in the \Gls{msbr}. Data from Table 3.1 and S.1 in \cite{robertson_conceptual_1971}. Ranged values collapsed to their average are denoted with a $^*$}
    \label{tab:hastelloy-n-ref}
    \begin{tabular}{|c|c|c|c|c|c|c|c|c|c|c|c|c|c|c|c|c|}
        \hline
        \ce{Ni} & \ce{Mo}$^*$ & \ce{Cr}$^*$ & \ce{Fe}$^*$ & \ce{C}$^*$ & \ce{Mn}$^*$ & \ce{Si} & \ce{W} & \ce{Al} & \ce{Ti}$^*$ & \ce{Cu} & \ce{Co} & \ce{P} & \ce{S} & \ce{B} & \ce{Hf}$^*$ & \ce{Nb}$^*$ \\
        \hline
        73.709 & 12 & 7 & 3 & 0.06 & 0.35 & 0.1 & 0.1 & 0.1 & 1.25 & 0.1 & 0.2 & 0.015 & 0.015 & 0.001 & 1 & 1\\
        %\hline
        %atom-\% & 76.72 & 7.64 & 8.224 & 3.282 & 0.305 & 0.389 & 0.218 & 0.033 & 0.226 & 1.595 & 0.096 & 0.207 & 0.03 & 0.029 & 0.006 & 0.342 & 0.658 \\
        \hline
    \end{tabular}
\end{table}

\begin{table}[htpb]
    \centering
    \caption{Mass-\% of elements in modified Hastelloy N used in the \Gls{msbr} model.}
    \label{tab:hastelloy-n-model}
    \begin{tabular}{|c|c|c|c|}
        \hline
        \ce{Ni} & \ce{Cr} & \ce{W} & \ce{Al} \\
        \hline
        67.7 & 7.0 & 25.0 & 0.3 \\
        \hline
    \end{tabular}
\end{table}

The Hastelloy N material in the CSG model uses the composition specified in
Table \ref{tab:hastelloy-n-model} and has a density of 8.671 
\unit{\gram\per\centi\metre\cubed}. The model material has a different composition
than the reference material because\ldots


\section{Reactor core}
\label{sec:msbr-core}
The \Gls{msbr} core is split into three distinct different zones; zone I, zone
II, and the reflector zone. 

\begin{figure}[htpb]
    \centering
    \subfloat[][]{
        \includegraphics[width=0.45\linewidth]{figs/ch4/msbr_section_ref.png}
        \label{fig:msbr_sec_ref}
    }
    \subfloat[][]{
        \includegraphics[width=0.35\linewidth]{figs/ch4/msbr_section_openmc.png}
        \label{fig:msbr_sec_model}
    }
    \caption[Detail views of MSBR core region]{
        \subref{fig:msbr_sec_ref} Detail view of \Gls{msbr} reference design.
        \subref{fig:msbr_sec_model} Detail view of \Gls{msbr} CSG model.}

    \label{fig:msbr-detail}
\end{figure}

\subsection{Zone I} Zone I is the central-most region of the core, and is 13\%
fuel salt by volume. Zone I is divided into three subzones: zone I-A, zone I-B,
and the control rod zone.

Zone I-A and I-B consist of 4-in. $\times$ 4-in. (10.16-cm $\times$ 10.16-cm)
square graphite elements. Elements in both zones have axial ribs protruding from
the faces of the square elements near the corners, as well as cylindrical
channels in the centers of the elements. The only dimensional difference between
the zone I-A and I-B elements is the diameter of the cylindrical channels.

The purpose of the ribs is to\ldots

The purpose of the fuel holes is to\ldots

The reason zone I is split into two elements is because\ldots

The central part of zone I consists of zone I-A elements, whereas the outer part
of zone I consists of zone I-B elements \cite{robertson_conceptual_1971}.
Robertson et al. does not specify exact distribution of zone I-A and I-B
elements, so in the CSG model we assume that zone I consists entirely of I-B
elements.

Figures \ref{fig:msbr-ia-xy}, \ref{fig:msbr-ia-lattice} show cross-sectional
views of the zone I-A elements. Figures \ref{fig:msbr-ib-xz-full},
\ref{fig:msbr-ib-xy}, and \ref{fig:msbr-ib-lattice} show cross-sectional views
of the zone I-B elements. Tables \ref{tab:zone-ia-ref-specs} and
\ref{tab:zone-ib-specs} specify the dimensions of the zone I-A and I-B elements,
respectively.

\begin{table}[htpb]
    \centering
    \caption{Reference Zone I-A dimensions}
    \label{tab:zone-ia-ref-specs}
    \begin{tabulary}{\linewidth}{|C|C|C|}
    \hline
    Quantity & Dimension [in] & Dimension [\unit{\centi\metre}]\\
    \hline
    Section A inner diameter & 0.6 & 1.524 \\
    \hline
    Section A Outer diameter & 3.698 & 9.39292 \\
    \hline
    Rib tip radius of curvature & 0.25 & 0.635 \\
    \hline
    Rib tip to element spacing & 0.302 & 0.7608 \\
    \hline
    Rib tip to element center & 1.315 & 3.3401 \\
    \hline
    Zone A height (bottom) & 9 & 22.86\\
    \hline
    Zone B height & 156 & 396.24 \\
    \hline
    Zone A height (top) & 7.5 & 19.05 \\
    \hline
    Conical part height & 2.75 & 6.985 \\
    \hline
    Hastelloy plug height & 2.875 & 7.3025 \\
    \hline
    Hastelloy plug diameter\footnote{the hastelloy N plug has the same diameter as the fuel hole in the I-A elements} & 0.6 & 1.524 \\
    \hline
    Top part height & 1.75 & 4.445 \\
    \hline
    Top part outer diameter & 1.75 & 4.445 \\
    \hline
    \end{tabulary}
\end{table}


\begin{table}[htpb]
    \centering
    \caption{Zone I-B dimensions}
    \label{tab:zone-ib-specs}
    \begin{tabulary}{\linewidth}{|C|CC|CC|}
    \hline
    Quantity & Reference Dimension [in] & Reference Dimension [\unit{\centi\metre}] & Model Dimension [in] & Model Dimension [\unit{\centi\metre}]\\
    \hline
    Section A inner diameter & 1.340 (Fig. \ref{fig:msbr_ib_bottom_ref}) or 1.347 (Fig. \ref{fig:msbr_ib_lattice_ref}) & 3.4036 or 3.42138 & 1.347 & 3.42138 \\
    \hline
    Section A Outer diameter & 3.900 & 9.9062 & 3.900 & 9.9062 \\
    \hline
    Rib tip radius of curvature & 0.25 & 0.635 & 0.263 & 0.66802\\
    \hline
    Rib tip to element spacing & 0.1 & 0.254 & 0.1 & 0.254\\
    \hline
    Rib tip to element center & 1.687 & 4.28498 & 1.687 & 4.28498\\
    \hline
    Zone A height (bottom) & 9 & 22.86 & 9 & 22.86\\
    \hline
    Zone B height & 156 & 396.24 & 156 & 396.24\\
    \hline
    Zone A height (top) & 7.5 & 19.05 & 7.5 & 19.05\\
    \hline
    Conical part height & 2.75 & 6.985 & 2.75 & 6.985\\
    \hline
    Hastelloy plug height & 2.875 & 7.3025 & 1.75 & 4.445 \\
    \hline
    Hastelloy plug diameter\footnote{the hastelloy N plug has the same diameter as the fuel holdein the I-B elements} & 1.340 (Fig. \ref{fig:msbr_ib_bottom_ref}) or 1.347 (Fig. \ref{fig:msbr_ib_lattice_ref}) & 3.4036 or 3.42138 & 1.347 & 3.42138 \\
    \hline
    Top part height & 1.75 & 4.445 & 1.75 & 4.445 \\
    \hline
    Top part outer diameter & 1.75 & 4.445 & 1.75 & 4.445\\
    \hline
    \end{tabulary}
\end{table}


\begin{figure}[htpb]
    \centering
    \subfloat[][]{
        \includegraphics[width=0.25\linewidth]{figs/ch4/zone_i_full_ref.png}
        \label{fig:msbr_i_full_ref}
    }
    \subfloat[][]{
        \includegraphics[width=0.17\linewidth]{figs/ch4/zone_ib_full_openmc.png}
        \label{fig:msbr_ib_full_model}
    }
    %\end{tabular}
    \caption[$xz$ cross sections of Zone I-B elements]{
        \subref{fig:msbr_i_full_ref} Reference Zone I element
        \subref{fig:msbr_ib_full_model} Model Zone I-B element
    }
    \label{fig:msbr-ib-xz-full}
\end{figure}

\begin{figure}[htpb]
    \centering
    \subfloat[][]{
        \includegraphics[width=0.3\linewidth]{figs/ch4/zone_ib_main_ref.png}
        \label{fig:msbr_ib_main_ref}
    }
    \subfloat[][]{
        \includegraphics[width=0.2\linewidth]{figs/ch4/zone_ib_main_openmc.png}
        \label{fig:msbr_ib_main_model}
    }
    \\
    \subfloat[][]{
        \includegraphics[width=0.3\linewidth]{figs/ch4/zone_ib_bottom_ref.png}
        \label{fig:msbr_ib_bottom_ref}
    }
    \subfloat[][]{
        \includegraphics[width=0.2\linewidth]{figs/ch4/zone_ib_bottom_openmc.png}
        \label{fig:msbr_ib_bottom_model}
    }
    \caption[$xy$ cross sections of Zone I-B elements]{
        \subref{fig:msbr_ib_main_ref} Reference Zone I-B element; Section B.
        \subref{fig:msbr_ib_main_model} Model Zone I-B element; Section B.
        \subref{fig:msbr_ib_bottom_ref} Reference Zone I-B element; Section A.
        \subref{fig:msbr_ib_bottom_model} Model Zone I-B element; Section A.
    }
    \label{fig:msbr-ib-xy}
\end{figure}

\begin{figure}[htpb]
    \centering
    \subfloat[][]{
        \includegraphics[width=0.25\linewidth]{figs/ch4/zone_ib_lattice_ref.png}
        \label{fig:msbr_ib_lattice_ref}
    }
    \subfloat[][]{
        \includegraphics[width=0.17\linewidth]{figs/ch4/zone_ib_lattice_openmc.png}
        \label{fig:msbr_ib_lattice_model}
    }
    %\end{tabular}
    \caption[Lattice view of Zone I-B elements]{
        \subref{fig:msbr_ib_lattice_ref} Reference Zone I-B lattice
        \subref{fig:msbr_ib_lattice_model} Model Zone I-B lattice 
    }
    \label{fig:msbr-ib-lattice}
\end{figure}


\begin{figure}[htpb]
    \centering
    \subfloat[][]{
        \includegraphics[width=0.3\linewidth]{figs/ch4/zone_ia_main_ref.png}
        \label{fig:msbr_ia_main_ref}
    }
    \subfloat[][]{
        \includegraphics[width=0.3\linewidth]{figs/ch4/zone_ia_bottom_ref.png}
        \label{fig:msbr_ia_bottom_ref}
    }
    \caption[$xy$ cross sections of Zone I-A elements]{
        \subref{fig:msbr_ia_main_ref} Reference Zone I-A element; Section B.
        \subref{fig:msbr_ia_bottom_ref} Reference Zone I-A element; Section A.
    }
    \label{fig:msbr-ia-xy}
\end{figure}

\begin{figure}[htpb]
    \centering
    \includegraphics[width=0.25\linewidth]{figs/ch4/zone_ia_lattice_ref.png}
    \caption{Lattice view of Zone I-A elements}
    \label{fig:msbr-ia-lattice}
\end{figure}



The control rod elements\ldots


\subsection{Zone II}

\begin{table}[htpb]
    \centering
    \caption{Zone II-A dimensions}
    \label{tab:zone-iia-specs}
    \begin{tabulary}{\linewidth}{|C|CC|CC|}
    \hline
    Quantity & Reference Dimension [in] & Reference Dimension [\unit{\centi\metre}] & Model Dimension [in] & Model Dimension [\unit{\centi\metre}]\\
    \hline
    Section A inner diameter & 0.5 & 1.27 & 0.5 & 1.27 \\
    \hline
    Section A Outer diameter & 2.875 & 7.3025 & 2.875 & 7.3025 \\
    \hline
    Section B inner diameter & 2.6 & 6.604 & 2.6 & 6.604 \\
    \hline
    Rib tip radius of curvature & 0.25 & 0.635 & 0.263 & 0.66802\\
    \hline
    Rib tip to element spacing & 0.1 & 0.254 & 0.1 & 0.254\\
    \hline
    Rib tip to element center & 1.687 & 4.28498 & 1.687 & 4.28498\\
    \hline
    Section B height & 172 & 436.88 & 172 & 436.88\\
    \hline
    Section A height & 2 & 5.08 & 2 & 5.08 \\
    \hline
    Cone height & 1 & 2.54 & 1 & 2.54 \\
    \hline
    Top part height & 3 & 7.62 & 3 & 7.62 \\
    \hline
    Top part outer radius & 1.75 & 4.445 & 1.75 & 4.445 \\
    \hline
    \end{tabulary}
\end{table}

\begin{figure}[htpb]
    \centering
    \subfloat[][]{
        \includegraphics[width=0.25\linewidth]{figs/ch4/zone_iia_full_ref.png}
        \label{fig:msbr_iia_full_ref}
    }
    \subfloat[][]{
        \includegraphics[width=0.17\linewidth]{figs/ch4/zone_iia_full_openmc.png}
        \label{fig:msbr_iia_full_model}
    }
    %\end{tabular}
    \caption[$xz$ cross sections of Zone II-A elements]{
        \subref{fig:msbr_iia_full_ref} Reference Zone II-A element
        \subref{fig:msbr_iia_full_model} Model Zone II-A element
    }
    \label{fig:msbr-iia-xz-full}
\end{figure}

\begin{figure}[htpb]
    \centering
    %\begin{tabular}{cc}
    \subfloat[][]{
        \includegraphics[width=0.3\linewidth]{figs/ch4/zone_iia_top_xy_ref.png}
        \label{fig:msbr_iia_top_ref}
    } 
    \subfloat[][]{
        \includegraphics[width=0.2\linewidth]{figs/ch4/zone_iia_top_xy2_openmc.png}
        \label{fig:msbr_iia_top_model}
    }
    \\
    \subfloat[][]{
        \includegraphics[width=0.35\linewidth]{figs/ch4/zone_iia_main_ref.png}
        \label{fig:msbr_iia_main_ref}
    } 
    \subfloat[][]{
        \includegraphics[width=0.15\linewidth]{figs/ch4/zone_iia_main_openmc.png}
        \label{fig:msbr_iia_main_model}
    }
    %\end{tabular}
    \caption[$xy$ cross sections of Zone II-A elements]{
        \subref{fig:msbr_iia_top_ref} Reference Zone II-A element; Section A.
        \subref{fig:msbr_iia_top_model} Model Zone II-A element; Section A.
        \subref{fig:msbr_iia_main_ref} Reference Zone II-A element; Section B.
        \subref{fig:msbr_iia_main_model} Model Zone II-A element; Section B.
    }
    \label{fig:msbr-iia-xy}
\end{figure}


\subsection{Reflectors}

\section{Vessel}
\label{sec:msbr-vessel}

\section{Reprocessing system}
\label{sec:msbr-reprocessing-system}

\section{Cross Section Data}

\section{Summary}
