\chapter{Molten Salt Reactor Modeling}%
\label{cha:msr_modeling}
%% lit review goes here
Much of our knowledge about \Gls{MSR}s come from experiements on a test reactor called the \Gls{MSRE} conducted at \Gls{ORNL} in the 1960s, which demonstrated the viability of the \Gls{MSR} concept for use in civillian power programs \cite{haubenreich_experience_1970} \cite{rosenthal_molten-salt_1970}.
%% Give brief description of the MSRE
%% cite a bunch of MSRE ORNL work.

The \Gls{MSRE} reactor to this day remains one of the few \Gls{MSR}s to operate.
%% Maybe briefly mention other historical MSRs?

As of writing this thesis, there are no \Gls{MSR}s currently in operation. Therefore, we must entirely rely on computational and/or surrogate models to further study \Gls{MSR} physics. The \Gls{MSRE} is a popular choice for computational models due to the availability of experimental data to compare results against. For example, Roelofs et al used the system thermal hydraulics code SPECTRA to model steady state parameters of the fuel-salt, \Gls{DNP} drift, and various fission product behaviors
in comparison with actual MSRE data \cite{roelofs_molten_2021}.  Podila et al performed a \Gls{CFD} simulation of the \Gls{MSBR} core to investigate the ability of \Gls{CFD} to predict 3D effects in this kind of reactor\cite{podila_cfd_2019}. 
%cite some MSRE studies
In addtion to the \Gls{MSREBR}, the \Gls{MSFR}(cite) and \Gls{MSBR}(cite), while conceptual, are well developed and have many associated studies. See Table () for a summary. These efforts illustrate that \Gls{MSR} modeling encompasses a wide range of physics domains.

Park et al perfomred a whole core analysis of the \Gls{MSBR} using MCNP6 with additional depletion and reprocessing using CINDER90 and a custom Python script \cite{park_whole_2015}.

\section{Modeling depletion in \Gls{MSR}s}
Recall in Sections
\ref{sec:molten_salt_reactors} and \ref{sec:msr_codes} we introduced the concepts of fuel depletion and removal and feed processes, and established the importance of modeling fuel depletion to \Gls{MSR} lisencing. Degletion codes in the past have shown good behavior when compared with experiment, although most of these models are . In M Table (table) summarizes some available software tools that can model depletion in \Gls{MSR}s.

%% Discuss briefly results from depletion papers

The to accurate \Gls{MSR} models.
A survey of some recent \Gls{MSR}s modeling efforts are summarized in Table (table?). 

%% talk about internal tools that natively support dpeletion (serpent, openmc, mention shift)

%% then talk about external tools (SaltProc, ChemTrition)

We are focused on OpenMC and Serpent because\ldots
%% talk about saltproc in it's current state, the gaps that exist
%% lead into discussion about why it's important to add openmc 
%% capabilities to SaltProc. Also mention the differnces in 
%% Serpent2 and OpenMC cross section handling and how this
%% adds capability to the SaltProc tool


%% Move this to the next chapter
\section{Sotware Overview and Development}
\label{sec:soft_dev}
A major component of this work was restructuring of the SaltProc code and implementation of OpenMC. In this section, I will provide a high level overview of my development process and go into detail where necessary. The release notes contain more details for those interested.
\subsection{SaltProc}%
\label{sub:saltproc}

SaltProc\cite{rykhlevskii_saltproc_2018} is an open source Python package that simulates on-line reprocessing via a batch-wise approach\footnote{Material is moved to or from the core at specific time intervals} in liquid-fueled \Gls{msr}s. More precisely, SaltProc manages material flows and separation processes on nuclides in the fuel. SaltProc relies on external codes to simulate fuel depletion.

The first version of SaltProc (v0.1) was a simple Python 2.7 package that used SERPENT 2 for the fuel depletion simulations. A single Python file contained all functions; separation processes applie dto  used an implicit 100\% efficiency \cite{rykhlevskii_advanced_2018}. The structure of SaltProc v0.1 did not lend itsef easily to development of more sophisticated treatments of reprocessing. This led to the release of Saltproc v0.2, in which the entire codebase was refactored into an object oriented
context in Python 3. The addition of new functionality in the \verb.Process. classes enabled more sophisticated treatment of online reprocessing \cite{rykhlevskii_fuel_2020} 

SaltProc v0.3 saw the implementation of processes for gas sparging and separation to more accurately simulate the MSBR, as well as additional refactoring to better follow OOP concepts.

\subsubsection{Preparing for OpenMC support}%
While v0.2 and v0.3 saw most of SaltProc refactored for OOP, before I could implement OpenMC support, I needed to resolve the following issues:
\begin{enumerate}
    \item Relocate several functions to have better separation of concerns
    \item Overhaul the SaltProc input file format giving users more control over their simulations
    \item Generalize docstrings\footnote{Since SaltProc was initally written as a script wrapped around Serpent2, many of the docstrings explicitly referenced Serpent2.}
    \item Improve function and variable names
\end{enumerate}
I implemented these changes as part of the 0.4.0 release. The changes in that release changed the API, making 0.4.0 incompatible with previous verions of SaltProc.
% consider talking about automation?

\section{OpenMC}%
\label{sub:openmc}

OpenMC \cite{romano_openmc_2015} is an open source Monte Carlo particle transport code. The \Gls{crpg} at \Gls{mit} started developing OpenMC back in 2011 with a focus on scalability for exascale computing. Since that time, developers new and old contributed features (cite?) and fixes to the tool expanding its scope and use cases. Notable features of OpenMC (as of version 0.12.1) are as follows \cite{homepage_openmc_2022}:
\begin{itemize}
    \item Support for fixed source, $k$-eigenvalue, and subcritical neutron multiplication cacluclations.
    \item Support for \Gls{csg} and \Gls{cad} geometry.
    \item Support for both continuous and multigroup transport calculations.
    \item Support for parallel execution via MPI and OpenMP.
    \item Geometry visualization through the Python API.
\end{itemize}
The tool is now quite mature and feature-rich, and is a legitmate alternative to it's closed-source counterparts in many cases.

Recently, depletion and photon transport were added by (who?)... The new depletion feature enables us to couple OpenMC to SaltProc and a fully open-source stack.

\subsection{Adding OpenMC to SaltProc}%
\label{sub:adding_openmc_to_saltproc}
