\chapter{Derivation of extraction efficiency equation}
\label{appex:extraction-efficiency}

Consider a material containing $m$ grams of element $X$. Suppose we apply
a process to this material, and let the extraction efficiency for element $X$ be
$\epsilon_{X}$. The mass of element $X$ remaining in the material after applying
the process, $m_{1}$ is
\begin{equation}
   m_{1} = m(1-\epsilon_{X}) 
\end{equation}

Suppose now that we apply the same process to the material a second time. The mass of element $X$ remaining in the material, $m_{2}$ is 
\begin{equation}
    m_{2} = m_{1}(1-\epsilon_{X}) = m(1-\epsilon_{X})^{2}
\end{equation}

It follow by induction that the leftover mass of element $X$ after applying this
process $n$ times is
\begin{equation}
    m_{n} = m(1-\epsilon_{X})^{n}
\end{equation}

Now, let $\delta \equiv \frac{m_{n}}{m}$, the fraction of original mass of
element $X$ remaining in the material. Suppose we pick a process that represents
a real chemical process with a cycle time\footnote{the amount of time it takes
to remove 100\% of the element} of $c_{X}$ for element $X$. Let $t$ be a value
of time between 0 and $c_{X}$. Then, at time $t$, the fraction of mass
of element $X$ remaining in the material is given by

\begin{equation}
    \delta = (1-\epsilon_{X})^{\frac{c_{X}}{t}}
\end{equation}

Consider the case where $t=c_{X}$. There is no value of $\epsilon_{X} < 1$ that
gives $\delta = 0$. It is also very difficult to separate 100\% of a substance
from a material in chemistry, so a natural solution arises in assigning a small,
nonzero value to $\delta$ as a constraint to obtain a separation efficiency.
Rearranging our equation, we get

\begin{equation}
    \epsilon_{X} = 1 - \delta^{\frac{t}{c_{X}}}
\end{equation}

If $t$ is greater than $c_{X}$, then the above equation no longer applies and we simply assign $\epsilon_{X} = 1$

\chapter{SaltProc input file structure}
\label{appex:input-files}
\definecolor{bg}{rgb}{0.95,0.95,0.95}

\begin{listing}[!ht]
    \begin{minted}[frame=single,
                   framesep=3mm,
                   tabsize=4,
                   bgcolor=bg,
                   fontsize=\footnotesize]{json}
    {
      "Path to Serpent executable": "sss2",
      "File containing processing system objects": "msbr_objects.json",
      "Graph file containing processing system structure": "msbr.dot",
      "User's Serpent input file with reactor model": "msbr.serpent",
      "Path output data storing folder": "../../saltproc/data/",
      "Output HDF5 database file name": "msbr_kl_100_saltproc.h5",
      "Number of neutrons per generation": 50,
      "Number of active generations": 20,
      "Number of inactive generations": 20,
      "Restart simulation from the step when it stopped?": false,
      "Geometry file/files to use in Serpent runs": "geometry/msbr_full.ini",
      "Switch to another geometry when keff drops below 1?": false,
      "Salt mass flow rate throughout reactor core (g/s)": 9920000,
      "Number of steps for constant power and depletion interval case": 12,
      "Depletion step interval or Cumulative time (end of step) (d)": 3,
      "Reactor power or power step list during depletion step (W)": 2250000000
    }
    \end{minted}
    \caption{\SaltProc v0.3.0 input file}
    \label{listing:1}
\end{listing}

\begin{listing}[!ht]
    \begin{minted}[frame=single,
                   framesep=3mm,
                   tabsize=4,
                   bgcolor=bg,
                   fontsize=\footnotesize]{json}
    {
       "proc_input_file": "msbr_objects.json",
       "dot_input_file": "msbr.dot",
       "output_path": "./data",
       "num_depsteps": 12,
       "depcode": {
           "codename": "serpent",
           "exec_path": "sss2",
           "template_inputfile_path": "./msbr.serpent",
           "iter_inputfile": "saltproc_serpent",
           "iter_matfile": "saltproc_mat",
           "npop": 50,
           "active_cycles": 20,
           "inactive_cycles": 20,
           "geo_file_paths": ["./geometry/msbr_full.ini"]
       },
       "simulation": {
           "sim_name": "msbr_example_simulation",
           "db_name": "msbr_kl_100_saltproc.h5",
           "restart_flag": false,
           "adjust_geo": false
       },
       "reactor": {
           "volume": 1.0,
           "mass_flowrate": 9920000,
           "power_levels": [ 2250000000 ],
           "dep_step_length_cumulative": [ 3 ]
       }
    }
    \end{minted}
    \caption{\SaltProc v0.4.0 input file}
    \label{listing:2}
\end{listing}

\begin{listing}[!ht]
    \begin{minted}[frame=single,
                   framesep=3mm,
                   tabsize=4,
                   bgcolor=bg,
                   fontsize=\footnotesize]{json}
    {
       "proc_input_file": "msbr_objects.json",
       "dot_input_file": "msbr.dot",
       "n_depletion_steps": 12,
       "depcode": {
           "codename": "serpent",
           "template_input_file_path": "msbr.serpent",
           "geo_file_paths": ["geometry/msbr_full.ini"]
       },
       "simulation": {
           "sim_name": "msbr_kl_100_simulation"
       },
       "reactor": {
           "volume": 1.0,
           "mass_flowrate": 9920000,
           "power_levels": [ 2250000000 ],
           "depletion_timesteps": [ 3 ],
           "timestep_units": "d"
       }
    }
    \end{minted}
    \caption{\SaltProc v0.5.0 input file}
    \label{listing:3}
\end{listing}
