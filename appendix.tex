\chapter{Appendix A: Derivation of extraction efficiency equation}
\label{appex:extraction-efficiency}

Consider a material containing $m$ grams of element $X$. Suppose we apply
a process to this material, and let the extraction efficiency for element $X$ be
$\epsilon_{X}$. The mass of element $X$ remaining in the material after applying
the process, $m_{1}$ is
\begin{equation}
   m_{1} = m(1-\epsilon_{X}) 
\end{equation}

Suppose now that we apply the same process to the material a second time. The mass of element $X$ remaining in the material, $m_{2}$ is 
\begin{equation}
    m_{2} = m_{1}(1-\epsilon_{X}) = m(1-\epsilon_{X})^{2}
\end{equation}

It follow by induction that the leftover mass of element $X$ after applying this
process $n$ times is
\begin{equation}
    m_{n} = m_(1-\epsilon_{X})^{n}
\end{equation}

Now, let $\delta \equiv \frac{m_{n}}{m}$, the fraction of original mass of
element $X$ remaining in the material. Suppose we pick a process that represents
a real chemical process with a cycle time\footnote{the amount of time it takes
to remove 100\% of the element} of $c_{X}$ for element $X$. Let $t$ be a value
of time between 0 and $c_{X}$. Then, at time $t$, the fraction of mass
of element $X$ remaining in the material is given by

\begin{equation}
    \delta = (1-\epsilon_{X})^{\frac{c_{X}}{t}}
\end{equation}

Consider the case where $t=c_{X}$. There is no value of $\epsilon_{X} < 1$ that
gives $\delta = 0$. It is also very difficult to seprate 100\% of a substance
from a material in chemistry, so a natural solution arises in assinging a small,
nonzero value to $\delta$ as a constraint to obtain a seprataion efficiency.
Rearranging our equation, we get

\begin{equation}
    \epsilon_{X} = 1 - \delta\frac{t}{c_{X}}
\end{equation}

If $t$ is greater than $c_{X}$, then the above equation no longer applies and we simply assign $\epsilon_{X} = 1$



