%% Package and Class "uiucthesis2014" for use with LaTeX2e.
\documentclass[edeposit,fullpage,hidelinks]{uiucthesis2018}


\usepackage[acronym,toc]{glossaries}
\usepackage{listings}
\usepackage{minted}
\usepackage{appendix}

\usepackage{xspace}
\usepackage{graphics}

\newcommand{\Cycamore}{\textsc{Cycamore}\xspace}
\newcommand{\Cyclus}{\textsc{Cyclus}\xspace}
\newcommand{\SaltProc}{\textsc{SaltProc}\xspace}
\newcommand{\OpenMC}{\textsc{OpenMC}\xspace}
\newcommand{\SerpentTWO}{\textsc{Serpent2}\xspace}
\newcommand{\ONIX}{\textsc{ONIX}\xspace}
\newcommand{\NJOYTWOONE}{\textsc{NJOY21}\xspace}
\newcommand{\ChemTriton}{\textsc{ChemTriton}\xspace}
\newcommand{\Shift}{\textsc{Shift}\xspace}
\newcommand{\SCALE}{\textsc{SCALE}\xspace}
\newcommand{\TRITON}{\textsc{TRITON}\xspace}
\newcommand{\ORIGEN}{\textsc{ORIGEN}\xspace}
\newcommand{\MCNPSIX}{\textsc{MCNP6}\xspace}
\newcommand{\CINDERNINETY}{\textsc{CINDER90}\xspace}
\newcommand{\ADDER}{\textsc{ADDER}\xspace}

\usepackage{placeins}
\usepackage{booktabs} % nice rules (thick lines) for tables
\usepackage{microtype} % improves typography for PDF

\usepackage[hyphens]{url}
\usepackage{hyperref}
\usepackage{subfig}
\usepackage{hhline}
\usepackage{amsmath}
\usepackage{color}
\usepackage{multirow}
\usepackage{siunitx} % typesetting for units
\usepackage{tabulary}
\usepackage[version=4]{mhchem}
\sisetup{
    input-decimal-markers = .,input-ignore = {,},table-number-alignment = right,
    group-separator={,}, group-four-digits = true
}
\usepackage{booktabs}
\newcommand\tab[1][1cm]{\hspace*{#1}}

\usepackage{threeparttable, tablefootnote}

%tikzpicture fit to page width
\usepackage{environ}
\makeatletter
\newsavebox{\measure@tikzpicture}
\NewEnviron{scaletikzpicturetowidth}[1]{%
  \def\tikz@width{#1}%
  \def\tikzscale{1}\begin{lrbox}{\measure@tikzpicture}%
  \BODY
  \end{lrbox}
  \pgfmathparse{#1/\wd\measure@tikzpicture}%
  \edef\tikzscale{\pgfmathresult}%
  \BODY
}

\usepackage{tabularx}
\newcolumntype{b}{>{\hsize=1.0\hsize}X}
\newcolumntype{q}{>{\hsize=0.5\hsize}X}
\newcolumntype{R}{>{\raggedleft\arraybackslash\hsize=0.5\hsize}X}
\newcolumntype{z}{>{\hsize=0.75\hsize}X}
\newcolumntype{s}{>{\hsize=.5\hsize}X}
\newcolumntype{m}{>{\hsize=.75\hsize}X}

\usepackage{cleveref}
\usepackage{datatool}
\usepackage[numbers]{natbib}
\usepackage{notoccite}


\usepackage{tikz}
\usetikzlibrary{positioning, arrows, decorations, shapes}

\usetikzlibrary{shapes.geometric,arrows}
\tikzstyle{process} = [rectangle, rounded corners, minimum width=2.5cm, minimum height=1cm,text centered, draw=black, fill=blue!30]

\tikzstyle{object} = [ellipse, rounded corners, minimum width=3cm, minimum height=1cm,text centered, draw=black, fill=green!30]
\tikzstyle{objectr} = [ellipse, rounded corners, minimum width=3cm, minimum height=1cm,text centered, draw=black, fill=red!30]

\tikzstyle{empty} =  [rectangle, rounded corners, minimum width=2.5cm, minimum height=0.7cm,text centered, draw=black, fill=white!30]
\tikzstyle{arrow} = [thick,->,>=stealth]


\title{Implementation and validation of \OpenMC depletion capabilities in \SaltProc}
\author{Oleksandr Redin Yardas}
\department{Nuclear, Plasma, Radiological Engineering}
\schools{B.A., Grinnell College, 2020}
\msthesis
\advisor{Madicken Munk}
\degreeyear{2023}
\committee{Dr. Madicken Munk, Advisor \\ Professor Tomasz Kozlowski}

\renewcommand{\lstlistingname}{Code}% Listing -> Algorithm

\begin{document}
\newacronym{crpg}{CRPG}{Computational Reactor Physics Group}
\newacronym{mit}{MIT}{Massachusets Institute of Technology}
\newacronym{csg}{CSG}{constructive solid geometry}
\newacronym{cad}{CAD}{computer aided design}
\newacronym{ghg}{GHG}{greenhouse gas}
\newacronym{gif}{GIF}{Generation IV International Forum}
\newacronym{msr}{MSR}{molten salt reactor}
\newacronym{nrc}{NRC}{Nuclear Regulatory Commission}
\newacronym{rnd}{R\&D}{research and development}
\newacronym{mns}{M\&S}{modeling and simulation}
\newacronym{ent}{E\&T}{education and training}
\newacronym{doene}{DOE-NE}{Department of Energy Office of Nuclear Energy}
\newacronym{cc}{CC}{closed code}
\newacronym{oss}{OSS}{open source software}
\newacronym{iaea}{IAEA}{International Atomic Energy Agency}
\newacronym{oncore}{ONCORE}{Open-source Nuclear COdes for REactor analysis}
\newacronym{snf}{SNF}{spent nuclear fuel}
\newacronym{doe}{DOE}{Department of Energy}

\maketitle

\frontmatter
%% Create an abstract that can also be used for the ProQuest abstract.
%% Note that ProQuest truncates their abstracts at 350 words.
\begin{abstract}
No operating molten salt reactors (MSRs) exist, so we must rely on computer simulations
to study this technology. Popular MSR designs to use in simulations
include the Molten Salt Breeder Reactor (MSBR), the Molten Salt Reactor Experiment
(MSRE), and the Molten Salt Fast Reactor (MSFR). A key process to
consider for MSRs is on-line fuel reprocessing and its effect on reactor
dynamics. \SaltProc is an open source interface code tool that simulates on-line
fuel reprocessing when coupled with a depletion-capable transport solver.
\SaltProc was intended to be usable with any depletion solver, but initially was written
with support for \SerpentTWO exclusively. In this work, I overhauled the \SaltProc code
to simplify coupling supports for depletion solvers and added support for \OpenMC.
To ensure that the code maintained existing functionality as well as extended it, I
validated the new functionality on a full-core model of the MSBR. In this work, I find
$k_\text{eff}$ between \OpenMC and \SerpentTWO differ by +35 pcm at beginning of life
and by 700 pcm at the end of the simulation after one year of operation. I also find
that the relative difference of most actinides is less than 2\%. The actinides
with smaller concentrations in the fuel tend to have larger relative differences
between the two codes. I find similar results for fission products, however, the 
relative difference is smaller, with most of the fission products having a
relative difference no greater than 0.6\%. There are some outliers in the results
with very high relative errors, however, these can be attributed to numerical errors.
This is significant as this makes \SaltProc the first completely open-source tool to
flexibly and accurately simulate on-line fuel reprocessing.
\end{abstract}

\chapter*{Acknowledgments}

This manuscript is the culmination of hundreds of hours of work, a substantial
amount of which is not my own. I would like to thank Dr. Madicken Munk and Dr.
Paul Romano for their guidance and mentorship, and for opening my eyes to the
complexity, frustrations, and joys of software development. I would also like to
thank Dr. Tomasz Kozlowski for his thorough feedback and critical perspective
that have vastly improved the quality of this manuscript. I would also like to
thank the Nuclear Regulatory Commission for funding this work through the NRC
Integrated University Grant Program Fellowship.

This research made use of Idaho National Laboratory computing resources which
are supported by the Office of Nuclear Energy of the U.S. Department of Energy
and the Nuclear Science User Facilities under Contract No. DE-AC07-05ID14517.

I would also like to thank my Advanced Reactors and Fuel Cycles peers,
particularly Amanda Bachmann, Samuel Dotson, and Luke Seifert for their code
reviews and knowledge on nuclear systems, Python, and molten salt reactors. And
to Andrei Rykhlevskii, thank you for creating \SaltProc and for your response to
my probing questions. To the NPRE Staff who run the show behind the scenes, you
are awesome and are the ones who make this whole show work. I would especially
like to thank Scott Dalby for always being a friendly face, and Kristie
Stramaski, for her reasoned and fast responses to my sometimes panicked emails.

In my personal life, I am extremely fortunate to have the boundless love and
support of my mother, Zanna, and father, David, and for that I have endless
gratitude. To my brother, Nico, and sister, Sophie, thank you for showing me
different ways of being. To my partner BJ and my friends in OAC, I am so
thankful to have met you all during my time in Illinois. You have made it all
the more bearable.

Finally, I would like to thank Dr. Kathryn Huff for allowing me to join ARFC and
start on this journey. While we haven't had much time to work together, I am
thankful for meeting her and reigniting my optimism for the future of nuclear
power.

%% The thesis format requires the Table of Contents to come
%% before any other major sections, all of these sections after
%% the Table of Contents must be listed therein (i.e., use \chapter,
%% not \chapter*).  Common sections to have between the Table of
%% Contents and the main text are:
%%
%% List of Tables
%% List of Figures
%% List Symbols and/or Abbreviations
%% etc.

\tableofcontents
\listoftables
\listoffigures

%% Create a List of Abbreviations. The left column
%% is 1 inch wide and left-justified
%\chapter{List of Abbreviations}
%\printglossaries
%% Create a List of Symbols. The left column
%% is 0.7 inch wide and centered

\pagebreak
\mainmatter

\chapter{Introduction}%
\label{cha:introduction}

\section{Motivation}%
\label{sec:motivation}

Increasing \Gls{ghg} emissions due to human activity since the Industrial Revolution drives observed increases in average global surface temperature via the greenouse effect\cite{mitchell_greenhouse_1989} \cite{paola_a_arias_2021_ts} (\textbf{global warming}).
The long term impacts of global warming...
Rapid decrease in global \Gls{ghg} emissions can dampen the impact of global warming. Transitioning from fossil-fuel-based electrictity production methods to zero-emission \footnote{zero-emissions during operation; there are lifecycle carbon costs associated with any form of power production} electrcity production methods (\textbf{decarbonization}) would contribute significantly to a global \Gls{ghg} emission reduction, and in turn dampen the effects of global
warming. Our three main options for decarbonization are renewables (solar, wind, geothermal), hydropower, and nuclear power. We will need all of these technologies working together if we want to sucessfully decarboinze, however each technology has a specifc use case where it works best; solar and wind are useful in responding to rapid fluctuations in electicity usage, wheras nuclear, hydropower, and geothermal are better suited to
providing a baseload level of electricity (cite). Geothermal and hydropower are limited by geographical features (geothermal acticivity and elevated water sources), making nuclear the most flexibile option for a future carbon-free baseload. Unfortunatley, nuclear power is a divisive technology(cite), with both legitimate technological and policy concerns.%consider expanding on this?
Luckily, many of these concerns have technological solutions. Therefore, we should maintian and develop nuclear power technologies so that they can contribute to decarbonization efforts.

\section{Current and future trajectories of nuclear power}%
\label{sec:current_and_future_trajectories_of_nuclear_power}
Nuclear power currently constitutes a fifth of US domestic electricty production, and half of US zero-carbon electricity production (cite EIA). Unfortunatley, our current nuclear fleet faces several threats to its long-term survival. The most relevant of these is the state of the electricity market: record-low natural gas prices, increases in subsidies for renewables, and general lack of similar compensation for
nuclear power plants make operating nuclear power plants financially unsustainable in the long term given the current market conditions\footnote{This is a generalization as this issue is incredibly complex and requires more space that I can give to it in this thesis to fully appreciate} \cite{szilard_economic_2016}. %perhaps mention all plants that have been shut down recently?

Assuming that we resolve the economic threats to our current nuclear fleet, they are -- like all things -- subject to aging and deterioration; at some point in the future, we will need to shut down and decommission them. As explained in the previous section, nuclear power will be a key player in any decarbonization strategy even though it is controversial. To both address this controversy and maintain nuclear power's postion in our decarbonization technology stack, we will need to develop and
implement new kinds of nuclear reactors that are more sustainable, economically competitive, safe, and reliable than
their predecessors.
%% I want to make this point differently... the conclusion should be that we need new nuclear power tech to ensure we can decarbonize, but I don't feel like the preceding statements smoothly come to that conclusion in the way I have preseneted them here
 
\section{Generation IV Reactors}%
\label{sec:generation_iv_reactors}

The first generation of nuclear power reactors includes the first prototpyes and early civil deployments in pursuit of cheap and bountiful energy. The second generation of reactors were built on top of this momentum, as well as (in some cases) in response to the 1973 oil crisis (cite). Following the well publicized and documented accidents at Three Mile Island (1976) and Chernobyl (1981) power plants, the third generation of nuclear power was developed with increased saftey and reactor lifetime
in mind. The fourth generation of nuclear power
reactors will need to rapidly respond to increasing electricity consumption and the need for decarbonzation. This means the fourth generation of nuclear reactors need to be able to be built quickly, have widespread use, all while maintaining or increasing fuel efficiency, economic competitivness, and saftey and proliferation resistance.

This is essentially the conclusion that the \Gls{gif} -- a "co-operative international endeavour seeking to develop the feasability and performance of routh generation nuclear systems" \cite{gif_homepage} -- came to in their 2002 roadmap \cite roadmap. This effort selected six reactor technologies in total. One of these technologies, the \Gls{msr}\footnote{for this thesis {\it Molten Salt Reactor} refers specifically to the reactor type where the fissile material is dissolved in in the salt
coolant}, is the central technology of this thesis.
%% why the msr??

The \Gls{msr} is so named as it is a nuclear reactor that uses a mixture of liquid-phase salts as a coolant. This has several benefits over solid fueled reactors, including increased fuel utilization \cite{}, potential in-core \Gls{snf} recylcing, and very strong passive saftey features (cite) \ldots \Gls{msr}s could also provide heat for industrial applications (typically a carbon intensive process) (cite). \Gls{}

In preparation for \Gls{msr}s and other GenIV reactor applications, regulators like the NRC have \ldots (non-lwr approach) 
  
\section{Objectives}%
\label{sec:objectives}

Currently available software tools are insufficient for creating a licensing application due to the greater physical complexities of \Gls{msr}s (cite).
The \Gls{doe-ne} has created a list of functional needs for modeling and simulation software prioritized by their importance to a licesing application for \Gls{msr}s.
Notably, fuel depletion is one of the most important functional needs. Several software projects exist that meet this functional need specifically in the context
of \Gls{msr}s:
%you'll need to read these papers to refresh your memory a bit
\begin{itemize}
    \item ChemTriton (cite chem triton paper)
    \item Serpent2 (cite auferio)
    \item SaltProc (cite Rykhlevskii)
\end{itemize}

Among these tools, only one of them (SaltProc) is open source. Open source nuclear software is\ldots (benfits of open source) 
Now, while SaltProc itself is open source, previously relied exclusivley on Serpent2 to perform depletion calculations.
This Master's thesis has two primary objectvies
\begin{enumerate}
    \item Add support for the open source monte carlo particle transport code OpenMC
    \item Test the new feature on a well studied MSR system.
\end{enumerate}


- several things threatening current fleet
  - age
  - low ng prices
  - public opinion
  - loss of building experience

- need for 
- current nuclear fleet is very old, nat gas prices
 -> assert: nuclear power can do one of two things
    1. Current tech -> replace large nat-gas power stations 
    2. "new" nuclear -> relpace smaller power stations + expand scope
        of nuclear power to more specialized use cases (space, remote communities, etc)
-> six gen4 reactors -> molten salt reactor one of these
-> NRC liscening approach for advacned reactors will be based on current R\&D, specifcially M\&S
-> M\&S tools that answer all relevant licesning questions for msr specifically do not exist
-> SaltProc aims to be a tool that can answer some of these questions, as well as a general purpose
    research tool for simulating MSRs

\section{Objectives}%
\label{sec:objectives}

 free and open source software provides lowest barrier of entry
 - saltproc built for use with serpent which defeats the purpose of being
    FOSS
 - Code needs to be restructured to enable easier development of interfaces to 
 open source MC transport codes
 - We then need to implement the necessary machinery to interface with a specific
    open source MC transport code (OpenMC)
 - Once implemented, we need to validate with a serpent-based case

\chapter{Molten Salt Reactor Modeling}%
\label{cha:msr_modeling}
%% lit review goes here
Much of our knowledge about \Gls{MSR}s come from experiements on a test reactor called the \Gls{MSRE} conducted at \Gls{ORNL} in the 1960s, which demonstrated the viability of the \Gls{MSR} concept for use in civillian power programs \cite{haubenreich_experience_1970} \cite{rosenthal_molten-salt_1970}.
%% Give brief description of the MSRE
%% cite a bunch of MSRE ORNL work.

The \Gls{MSRE} reactor to this day remains one of the few \Gls{MSR}s to operate.
%% Maybe briefly mention other historical MSRs?

As of writing this thesis, there are no \Gls{MSR}s currently in operation. Therefore, we must entirely rely on computational and/or surrogate models to further study \Gls{MSR} physics. The \Gls{MSRE} is a popular choice for computational models due to the availability of experimental data to compare results against. For example, Roelofs et al used the system thermal hydraulics code SPECTRA to model steady state parameters of the fuel-salt, \Gls{DNP} drift, and various fission product behaviors
in comparison with actual MSRE data \cite{roelofs_molten_2021}.  Podila et al performed a \Gls{CFD} simulation of the \Gls{MSBR} core to investigate the ability of \Gls{CFD} to predict 3D effects in this kind of reactor\cite{podila_cfd_2019}. 
%cite some MSRE studies
In addtion to the \Gls{MSREBR}, the \Gls{MSFR}(cite) and \Gls{MSBR}(cite), while conceptual, are well developed and have many associated studies. See Table () for a summary. These efforts illustrate that \Gls{MSR} modeling encompasses a wide range of physics domains.

Park et al perfomred a whole core analysis of the \Gls{MSBR} using MCNP6 with additional depletion and reprocessing using CINDER90 and a custom Python script \cite{park_whole_2015}.

\section{Modeling depletion in \Gls{MSR}s}
Recall in Sections
\ref{sec:molten_salt_reactors} and \ref{sec:msr_codes} we introduced the concepts of fuel depletion and removal and feed processes, and established the importance of modeling fuel depletion to \Gls{MSR} lisencing. Degletion codes in the past have shown good behavior when compared with experiment, although most of these models are . In M Table (table) summarizes some available software tools that can model depletion in \Gls{MSR}s.

%% Discuss briefly results from depletion papers

The to accurate \Gls{MSR} models.
A survey of some recent \Gls{MSR}s modeling efforts are summarized in Table (table?). 

%% talk about internal tools that natively support dpeletion (serpent, openmc, mention shift)

%% then talk about external tools (SaltProc, ChemTrition)

We are focused on OpenMC and Serpent because\ldots
%% talk about saltproc in it's current state, the gaps that exist
%% lead into discussion about why it's important to add openmc 
%% capabilities to SaltProc. Also mention the differnces in 
%% Serpent2 and OpenMC cross section handling and how this
%% adds capability to the SaltProc tool


%% Move this to the next chapter
\section{Sotware Overview and Development}
\label{sec:soft_dev}
A major component of this work was restructuring of the SaltProc code and implementation of OpenMC. In this section, I will provide a high level overview of my development process and go into detail where necessary. The release notes contain more details for those interested.
\subsection{SaltProc}%
\label{sub:saltproc}

SaltProc\cite{rykhlevskii_saltproc_2018} is an open source Python package that simulates on-line reprocessing via a batch-wise approach\footnote{Material is moved to or from the core at specific time intervals} in liquid-fueled \Gls{msr}s. More precisely, SaltProc manages material flows and separation processes on nuclides in the fuel. SaltProc relies on external codes to simulate fuel depletion.

The first version of SaltProc (v0.1) was a simple Python 2.7 package that used SERPENT 2 for the fuel depletion simulations. A single Python file contained all functions; separation processes applie dto  used an implicit 100\% efficiency \cite{rykhlevskii_advanced_2018}. The structure of SaltProc v0.1 did not lend itsef easily to development of more sophisticated treatments of reprocessing. This led to the release of Saltproc v0.2, in which the entire codebase was refactored into an object oriented
context in Python 3. The addition of new functionality in the \verb.Process. classes enabled more sophisticated treatment of online reprocessing \cite{rykhlevskii_fuel_2020} 

SaltProc v0.3 saw the implementation of processes for gas sparging and separation to more accurately simulate the MSBR, as well as additional refactoring to better follow OOP concepts.

\subsubsection{Preparing for OpenMC support}%
While v0.2 and v0.3 saw most of SaltProc refactored for OOP, before I could implement OpenMC support, I needed to resolve the following issues:
\begin{enumerate}
    \item Relocate several functions to have better separation of concerns
    \item Overhaul the SaltProc input file format giving users more control over their simulations
    \item Generalize docstrings\footnote{Since SaltProc was initally written as a script wrapped around Serpent2, many of the docstrings explicitly referenced Serpent2.}
    \item Improve function and variable names
\end{enumerate}
I implemented these changes as part of the 0.4.0 release. The changes in that release changed the API, making 0.4.0 incompatible with previous verions of SaltProc.
% consider talking about automation?

\section{OpenMC}%
\label{sub:openmc}

OpenMC \cite{romano_openmc_2015} is an open source Monte Carlo particle transport code. The \Gls{crpg} at \Gls{mit} started developing OpenMC back in 2011 with a focus on scalability for exascale computing. Since that time, developers new and old contributed features (cite?) and fixes to the tool expanding its scope and use cases. Notable features of OpenMC (as of version 0.12.1) are as follows \cite{homepage_openmc_2022}:
\begin{itemize}
    \item Support for fixed source, $k$-eigenvalue, and subcritical neutron multiplication cacluclations.
    \item Support for \Gls{csg} and \Gls{cad} geometry.
    \item Support for both continuous and multigroup transport calculations.
    \item Support for parallel execution via MPI and OpenMP.
    \item Geometry visualization through the Python API.
\end{itemize}
The tool is now quite mature and feature-rich, and is a legitmate alternative to it's closed-source counterparts in many cases.

Recently, depletion and photon transport were added by (who?)... The new depletion feature enables us to couple OpenMC to SaltProc and a fully open-source stack.

\subsection{Adding OpenMC to SaltProc}%
\label{sub:adding_openmc_to_saltproc}

\chapter{Software Description}
Development on \SaltProc to enable use with \OpenMC consitutes a major portion
of this thesis. In this chapter, I will provide a high level overview
of my development process.\footnote{The release notes
contain more details for those interested} I will also provide an updated
description of the code structure using the new API. 

\section{SaltProc development history}%
\label{sub:saltproc-hisory}

\SaltProc\cite{rykhlevskii_saltproc_2018} is an open source Python package that
simulates on-line reprocessing via a batch-wise approach\footnote{Material is
moved to or from the core at specific time intervals} in liquid-fueled
\Gls{msr}s. More precisely, \SaltProc manages material flows and separation
processes on nuclides in the fuel. \SaltProc relies on external codes to simulate
fuel depletion.

The first version of \SaltProc (v0.1) was a simple Python 2.7 package that used
\SerpentTWO for the fuel depletion simulations. A single Python file contained
all functions; separation processes applied to materials used an implicit 100\%
extraction efficiency at hardcoded cycle times\cite{rykhlevskii_advanced_2018}.
The structure of \SaltProc v0.1 did not lend itsef easily to development of more
sophisticated treatments of reprocessing. This led to the release of \SaltProc
v0.2, in which the entire codebase was refactored into an object oriented
context in Python 3. The addition of new functionality in the \verb.Process.
classes enabled more sophisticated treatment of online reprocessing using user
defined extraction efficiencies and removed the cycle time functionality\cite{rykhlevskii_fuel_2020}.

\SaltProc v0.3 saw the implementation of processes for gas sparging and
separation to more accurately simulate the \gls{msbr}, as well as additional
refactoring to better follow OOP concepts.

While v0.2 and v0.3 saw most of \SaltProc refactored for OOP, before I could
implement \OpenMC support, I needed to resolve the following issues:
\begin{enumerate}
    \item Relocate several functions to have better separation of concerns
    \item Overhaul the \SaltProc input file format giving users more control over their simulations
    \item Generalize docstrings\footnote{Since \SaltProc was initally written as a script wrapped around Serpent2, many of the docstrings explicitly referenced Serpent2.}
    \item Improve function and variable names
\end{enumerate}
I implemented these changes as part of the 0.4.0 release. The changes in that
release changed the API, making 0.4.0 incompatible with previous verions of \SaltProc.

\SaltProc v0.5.0 added full support for \OpenMC, overhauled the test suite, and included additional API changes.

% consider talking about automation?

\section{OpenMC}%
\label{sub:openmc}

OpenMC \cite{romano_openmc_2015} is an open source Monte Carlo particle
transport code. The \Gls{crpg} at \Gls{mit} started developing OpenMC back in
2011 with a focus on scalability for exascale computing. Since that time,
developers new and old contributed features (cite?) and fixes to the tool
expanding its scope and use cases. Notable features of OpenMC (as of version
0.13.1) are as follows \cite{homepage_openmc_2022}:
\begin{itemize}
    \item Support for fixed source, $k$-eigenvalue, and subcritical neutron multiplication cacluclations.
    \item Support for \Gls{csg} and \Gls{cad} geometry.
    \item Support for both continuous and multigroup transport calculations.
    \item Support for parallel execution via MPI and OpenMP.
    \item Geometry visualization through the Python API.
    \item Transport-coupled and transport-independent depletion
\end{itemize}

\section{SaltProc v0.5.0}
\label{sec:saltproc-detail}

\subsection{Feeds and Separations}
\label{sub:feeds-separations}
\SaltProc models separations and feeds using three different data structures:


\paragraph{Material flows}
    A material flow represents a material with a given
    volume, density, and temperature, and nuclide composition.
    It can also include information such as void fraction. \verb.Materialflow.
    objects contain the prior mentioned quantities as attributes, as well as
    methods to perform the following:
    \begin{itemize}
        \item Add two \verb.Materialflow. objects together
        \item Multiply a \verb.Materialflow. object by a constant
    \end{itemize}
    \verb.Materialflow. objects are used to store information about materials
    in reprocessing as well as materials being fed into the system.

\paragraph{Processes}
    A process is an abstraction of chemical separation, extraction, or some
    other removal. Two processes commonly used in \Gls{msr}s include
    extracting metals using gas sparging and filtering. In \SaltProc,
    \verb.Process. objects contain data and functions to perform their
    associated processing task. At a minimum, a \verb.Process. object includes
    the following
    \begin{itemize}
        \item A mass flowrate, $\dot{m}$, that specifies the mass of fuel salt a process can operate on per unit time
        \item An extraction efficiency, $\epsilon$, for target element(s). This can be a constant value or a function.
        \item A method to apply the process on a \verb.Materialflow. object. 
    \end{itemize}

\paragraph{Graphs}
    A graph is a mathematical object that connects {\it nodes}\footnote{the
    terms verices and points are also used} with {\it edges}. More formally,
    a graph is a pair of sets, $(V, E)$, where
    \begin{itemize}
        \item $V$ is a set whose elements are called nodes
        \item $E$ is a set whose elements are pairs of elements in $V$
    \end{itemize}
    A directed graph is a graph where the elements are ordered pairs of elements
    in $V$. This gives the edges a direction. \SaltProc uses directed graphs to
    model the order and path in which processes operate on materials.
        
Processes and feesd are defined in one JSON input file, and the graph linking
processes is  defind in a DOT file. The process graph must be directed and
acyclic in order to work with \SaltProc v0.5.0. At runtime, \SaltProc reads this
input file to create \verb.Process. objects for each item in the file. Every
process file must have at lest a \verb.core_outlet. and \verb.core_inlet.
Process.

\subsection{Material reprocessing}
Recall that \SaltProc uses a {\it batchwise} reprocessing scheme.

Let $\mathbf{n}(t)^{j}$ denote the nuclide mass vector for depletable material
$j$ as function of time. For each depletable material $j$, the depletion solver numerically
integrates the equation

\begin{equation}
    \frac{d\mathbf{n}^{j}(t)}{dt} = \mathbf{A}(\mathbf{n}^{j}(t), t)
\end{equation}

from time $i$ to time $i+1$ to get $\mathbf{n}^{j}(i+1)$, where $A$ is the
depletion matrix. The specific details of integration are alredy covered in
numerous other works (cite some here). This is sequence is called a {\bf depletion
step}.

Now, let the mass and volume of depletable material $j$ be
$m^{j}$ and $V^{j}$ respectively. Let the mass flowrate of process $p$ be
$\dot{m}_{p}$. At the end of each depletion step, \SaltProc constructs process
paths from the process graph defined in the DOT file, and sequentially applies
each proceess $p$ in each path $r$ to the relevant materials to obtain thru
and waste streams for each material. \SaltProc tracks the mass and nuclide vector for both thru and waste streams. For thru
streams, \SaltProc also tracks the volume and mass flowrate. For every node
$p\in[0,l]$ where $0$ represents the core outlet and $l$ represents the core
inlet, in the path $r$, for the thru streams we have

% still need burnup, temp, density, void frac
\begin{equation}
    \mathbf{n}^{j}_{\text{thru, }p,r} = \mathbf{n}^{j}_{\text{thru, }p-1,r} (1 - \mathbf{\epsilon}^{j}_{p,r})
\end{equation}
\begin{equation}
    m^{j}_{\text{thru, } p,r} = \alpha_{p} m^{j}_{\text{thru, }p-1,r} - m^{j}_{\text{waste, }p,r}
\end{equation}
\begin{equation}
    V^{j}_{\text{thru, }p,r} = \alpha_{p}V^{j}_{\text{thru, }p-1,r}
\end{equation}
where 
\begin{equation}
    \alpha_{p,r} = \frac{\dot{m}_{p,r}}{\dot{m}_{\text{outlet}}}
\end{equation}
and the inital conditions are 
\begin{equation}
    \mathbf{n}^{j}_{\text{thru, }0,r} = \mathbf{n}^{j}(i+1)
\end{equation}
\begin{equation}
    m^{j}_{\text{thru, }0,r} = \rho^{j}(i+1)V^{j}(i+1)
\end{equation}
\begin{equation}
    V^{j}_{\text{thru, }0,r} = V^{j}(i+1)
\end{equation}
Similarly, for the waste streams, we have
\begin{equation}
    \mathbf{n}^{j}_{\text{waste, }p,r} = \mathbf{n}^{j}_{\text{thru, }p-1,r} \cdot \mathbf{\epsilon}^{j}_{p,r}
\end{equation}
\begin{equation}
    m^{j}_{\text{waste, }p,r} = \alpha_{p,r} m^{j}_{\text{thru, }p-1,r} \langle\mathbf{1},\mathbf{n}^{j}_{\text{waste, }p,r}\rangle
\end{equation}
\SaltProc does not currently track the volume and mass flowrate of waste streams.

After the recursive computation, \SaltProc sums thru and waste streams over all
paths to get the total thru stream at the inlet, and the total waste stream at
the inlet which represents all material removed during reprocessing.
\begin{equation}
    \mathbf{n}^{j}_{\text{thru, inlet, net}} = \frac{\sum_{r} m^{j}_{\text{thru, inlet, }r} \mathbf{n}^{j}_{\text{thru, inlet, }r}}{\sum_{r} m^{j}_{\text{thru, inlet, }r}}
\end{equation}
\begin{equation}
    \mathbf{n}^{j}_{\text{waste, inlet, net}} = \frac{\sum_{r} m^{j}_{\text{waste, inlet, }r} \mathbf{n}^{j}_{\text{waste, inlet, }r}}{m^{j}_{\text{waste, inlet, }r}}
\end{equation}

Before running the next depletion step, for any material that has an associated
feed material defined, \SaltProc will add an amount of the feed material
equivalent to the removed mass so that the mass of fuel salt undergoing depletion
remains constant.
For feed $j'$ corresponding to material $j$, we have
\begin{equation}
    \mathbf{n}^{j}_\text{filled} = \frac{m^{j}_{\text{thru, }l}\mathbf{n}^{j}_{\text{thru, inlet, net}} +  m^{j}_{\text{removed}}\mathbf{n}^{j'}}{m^{j}_{\text{thru,}0}}
\end{equation}

where 
\begin{equation}
    m^{j}_{\text{removed}} = m^{j}_{\text{thru,}0} - m^{j}_{\text{thru, } l}
\end{equation}

\chapter{Model Description}
\label{ch:chapter4}
As mentioned in Chapter 1, I am using a model of the \Gls{msbr}
\cite{robertson_conceptual_1971} to verify my \OpenMC implementation in
\SaltProc.

I picked the \Gls{msbr} because\ldots 

The \Gls{msbr} design is the result of a design study of a single-fluid
\Gls{msr} following the success of the \Gls{msre}
\cite{haubenreich_experience_1970}\cite{rosenthal_molten-salt_1970}.
I will only describe the following reactor systems that are relevant to
my validation study\footnote{A complete description of the entire \Gls{msbr}
system can be found in \cite{robertson_conceptual_1971}}: the fuel salt, the
reactor core, and the salt reprocessing system.

\begin{figure}[htpb] 
    \label{fig:msbr-overview}
    \centering
    \subfloat[][]{
        \includegraphics[width=0.5\linewidth]{figs/ch4/msbr_full_xy_ref.png}
        \label{fig:msbr_ref_xy}
    }
    \subfloat[][]{
        \includegraphics[width=0.5\linewidth]{figs/ch4/msbr_full_xy_openmc.png}
        \label{fig:msbr_model_xy}
    }
    \\
    \subfloat[][]{
        \includegraphics[width=0.5\linewidth]{figs/ch4/msbr_full_xz_ref.png}
        \label{fig:msbr_ref_xz}
    }
    \subfloat[][]{
        \includegraphics[width=0.5\linewidth]{figs/ch4/msbr_full_xz_openmc.png}
        \label{fig:msbr_model_xz}
    }
    \caption[Full core views of MSBR reference design and virtual model]{
        \subref{fig:msbr_ref_xy} Top down view of \Gls{msbr} reference design.
        \subref{fig:msbr_model_xy} Top down view of \Gls{msbr} CSG model.
        \subref{fig:msbr_ref_xz} Top down view of \Gls{msbr} reference design.
        \subref{fig:msbr_model_xz} Top down view of \Gls{msbr} CSG model.
    }
\end{figure}

As seen in Figure \ref{fig:msbr-overview} \OpenMC and \SerpentTWO \Gls{msbr}
models of reproduce these systems with several approixmations. I will
describe each reactor system, as well as any relavant changes or
approximations made in the model.

\section{Materials}
\label{sec:msbr-materials}

\subsection{Fuel salt}
\label{sub:msbr-fuel-salt}
Table S.1 in \cite{robertson_conceptual_1971} specifies the fuel salt
composition used in the \Gls{msbr}:
\ce{LiF}-\ce{Be}\ce{F_2}-\ce{Th}\ce{F_4}-\ce{U}\ce{F_4} at a
concentration of 71.7-16-12-0.3 mole-\%\footnote{In Rykhlevskii's thesis
\cite{rykhlevskii_fuel_2020}, he stated the mole-\% to be 71.75-16-12-0.25. I
have been unable to find this composition in Robertson et al.
\cite{robertson_conceptual_1971}}. The lithium used in the fuel salt is
enriched to 99.995\% \ce{^{7}Li}. This is because \ce{^{6}Li} is a strong
neutron absorber and produces tritium in the absorption reaction. The atom-\%
for each nuclide is given in Table \ref{tab:msbr_fuel_salt}\footnote{Most of the
discussion of the fuel salt compositon in Robertson el al
\cite{robertson_conceptual_1971} specify elemental version of the nuclides in
Table \ref{tab:msbr_fuel_salt}. I have specific specific nuclides for the
following reasons: (1) both \ce{F} and \ce{Be} have only one stable isotope, (2)
the fuel salt recieves initial fissile loading from \ce{^{233}U} or
\ce{^{235}U}, and (3) the fuel salt recieves its fertile loading from
\ce{^{232}Th}}.

\begin{table}[htpb] 
    \centering 
    \caption{\Gls{msbr} fuel salt specifications}
    \label{tab:msbr_fuel_salt}
    \begin{tabular}{|c|c|c|c|c|c|} 
        \hline
        \ce{^{6}Li} & \ce{^{7}Li} & \ce{^{19}F} & \ce{^{9}Be} & \ce{^{232}Th} & \ce{^{233}U}\\
        \hline 
        1.4357925 & 34.4142075 & 56.356 & 5.3 & 2.4 & 0.06 \\
        \hline
    \end{tabular}
\end{table}

The density of the fuel salt is given by a function\footnote{This function 
comes from an earlier report on the Molten-Salt Reactor Program
\cite{rosenthal_molten-salt-ornl_1970}.} of temperature in \unit{\celsius} in Table
S.1 in Robertson et al. \cite{robertson_conceptual_1971}:
\begin{equation}
    \rho = 3.752 - 6.68\cdot 10^{-4} \cdot T \quad \unit{\gram\per\square  \centi\meter}
\end{equation}

The temperature of the fuel salt flowing into the core at the inlet at the
bottom of the reactor is 1050\unit{\degree}F (565.5556\unit{\celsius}, 838.7056
\unit{\kelvin}), and the temperature of the fuel salt flowing out of the core at
the outlet is approximately 1300\unit{\degree}F (704.4444\unit{\celsius},
977.5944 \unit{\kelvin})\cite{robertson_conceptual_1971}. The average
temperature of the salt over the core inlets and outlets is then 1175
\unit{\degree}F (635\unit{\celsius}, 908.15 \unit{\kelvin}). While the
various solid components of the core are at a slightly higer temperature on
average\footnote{see figure 3.29 in \cite{robertson_conceptual_1971}}, for
simplicity, I set the evaluated temperature of all materials to 900
\unit{\kelvin} consistent cross-section selection between the \OpenMC and
\SerpentTWO depletion steps. At this temperature, the density of the fuel salt
is 3.3332642 \unit{\gram\per\centi\metre\cubed}.

In the virtual model, the fuel salt material uses the composition specified in
Table \ref{tab:msbr_fuel_salt} and has a density of 3.3332642
\unit{\gram\per\centi\metre\cubed}.

\subsection{Graphite}
\label{sub:graphite}

For a detailed description of the reactor graphite used in the \Gls{msbr}, see
Section 3.2.3 in \cite{robertson_conceptual_1971}. At 70\unit{\degree}F (294.3
\unit{\kelvin}), the \Gls{msbr} graphite has a density of 1843
\unit{\kilo\gram\per\cubic\metre}. This is the only density specification
for graphite that I was able to find in Robertson et al.
\cite{robertson_conceptual_1971}

In the virtual model, the graphite material uses elemental carbon and has a
density of 1.843\unit{\gram\per\centi\metre\cubed}.

\subsection{Modified Hastelloy N}
\label{sub:hastelloy}
Hastelloy N is an alloy developed at \Gls{ornl} during the Molten-Salt Reactor
Program as a structural material that could maintain structural stability while
in contact with the corrosive and high temperature molten salt fuel while also
being under irradation for a long period of time.

The \Gls{msbr} used a modified version of Hastelloy N designed to improve
embrittlement resistance and weldibility \cite{robertson_conceptual_1971}.
The \Gls{msbr} uses modified Hastelloy N on all nearly all salt-facing
components included in the virtual model.

Modified Hastelloy N has a density of 8671 \unit{\kilo\gram\per\cubic\metre} at
1300\unit{\degree}F (704.4444\unit{\celsius}, 977.5944 \unit{\kelvin})
\cite{robertson_conceptual_1971}. The elemental composition of modified
Hastelloy N and their amounts in mass-\% are in Table \ref{tab:hastelloy-n}.

\begin{table}[htpb]
    \centering
    \caption{Mass-\% of elements in modified Hastelloy N used in the \Gls{msbr}. Data from Table 3.1 and S.1 in \cite{robertson_conceptual_1971}. Ranged values are collapsed to their average.}
    \label{tab:hastelloy-n}
    \begin{tabular}{|c|c|c|c|c|c|c|c|c|c|c|c|c|c|c|c|c|}
        \hline
        \ce{Ni} & \ce{Mo} & \ce{Cr} & \ce{Fe} & \ce{C} & \ce{Mn} & \ce{Si} & \ce{W} & \ce{Al} & \ce{Ti} & \ce{Cu} & \ce{Co} & \ce{P} & \ce{S} & \ce{B} & \ce{Hf} & \ce{Nb} \\
        \hline
        73.703 & 12 & 7 & 3 & 0.06 & 0.35 & 0.1 & 0.1 & 0.1 & 1.25 & 0.1 & 0.2 & 0.015 & 0.015 & 0.001 & 1 & 1\\
        \hline
    \end{tabular}
\end{table}

The Hastelloy N material in the virtual model uses the composition specified in
Table \ref{tab:hastelloy-n} and has a density of 8.671
\unit{\gram\per\centi\metre\cubed}.


\section{Reactor core}
\label{sec:msbr-core}
The \Gls{msbr} core is split into three distinct different zones; zone I, zone
II, and the reflector zone.

\subsection{Zone I} Zone I is the central-most region of the core, and is 13\%
fuel salt by volume. Zone I is divided into three subzones: zone I-A, zone I-B,
and the control rod zone. (more about these zones)

\subsection{Zone II}


\subsection{Reflectors}

\section{Vessel}
\label{sec:msbr-vessel}

\section{Reprocessing system}
\label{sec:msbr-reprocessing-system}

\section{Cross Section Data}

\section{Summary}

\chapter{Results}
\label{ch:chapter5}

My  

\section{Platform and Software}
\label{sub:platform-software}
I ran the \OpenMC simulation on Sawtooth, a supercomputer at
\Gls{inl}. On Sawtooth, the compute nodes have 2 Intel Xeon 8268 CPUs with 24 cores per
CPU. Hyper-threading is disabled on compute nodes. Each node has 192 GB of RAM.
The login nodes have the same specifications. See Table \ref{tab:sawtooth-params}
for the full specifications of the machine. To avoid any potential licensing
issues with getting \SerpentTWO on Sawtooth, I performed the \SerpentTWO runs
on my office computer, a Dell Precision 3430 Workstation, using 12 OpenMP threads.

\begin{table}[htpb] 
    \centering 
    \caption{Sawtooth job parameters for 100k particle \OpenMC run}
    \label{tab:sawtooth-params}
    \begin{tabular}{|c|c|} 
        \hline
        Quantity & Value\\
        \hline
        Nodes & 92 \\
        \hline
        MPI processes per node & 6 \\
        \hline
        Threads per core & 1 \\
        \hline
        OpenMP threads per MPI process & 8 \\
        \hline
    \end{tabular}
\end{table}
% make this subsection a table

\subsection{Simulation Design and Parameters}
\label{sub:simulation-parameters}

The purpose of these simulations is to test the accuracy of
the implementation of \OpenMC support, so we do not need to do as detailed an
analysis as Ryhlevskii did on finding the equilibrium state. A moderate amount
of timesteps will suffice. I decided to use the same 3 day timesteps as Ryhklevskii
\cite{rykhlevskii_modeling_2019} (as discussed in Section \ref{sub:reprocessing-system-model})
for a year's worth of runtime. Table \ref{tab:saltproc-params} summarizes the simulation
settings used for both the \SerpentTWO and \OpenMC simulations. While I had intended
to use interpolated fission product yields for both the \OpenMC and \SerpentTWO simulations,
I realized very late into the process of writing this manuscript that I had forgotten to set
that option in \OpenMC. This may be one of the reasons some results have larger errors than expected.
 
\begin{table}[htpb] 
    \centering 
    \caption{Neutronics and Depletion parameters for \SaltProc}
    \label{tab:saltproc-params}
    \begin{tabular}{|c|c|} 
        \hline
        Batches & 200 \\
        \hline
        Inactive batches & 80 \\
        \hline
        Particles per batch & 1e6 \\
        \hline
        Power [W] & 2.25e9 \\
        \hline
        Depletion steps & 122 \\
        \hline
        Depletion step length [days] & 3 \\
        \hline
        Depletion equation solver & IPF CRAM 48 \\
        \hline
        Time integration method & Euler's Method \\
        \hline
    \end{tabular}
\end{table}

\subsection{Data}
\label{sub:results-xs-data}

I used neutron reaction cross sections from the {\bf endf71x} \cite{conlin_continuous_2013} library and
thermal scattering cross sections {\bf ENDF70SaB} \cite{trellue_release_2008} library
for both the \SerpentTWO and \OpenMC simulations. While the
{\bf endf71x} library is based on the evaluated neutron reaction data from the ENDF/B-VII.1 library \cite{chadwick_endf/b-vii.1_2011},
the {\bf ENDF70SaB} library is based on evaluated thermal scattering data from the ENDF/B-VII.0 library \cite{chadwick_endfb-vii0_2006}.
Fortunately, the thermal scattering data in the ENDF/B-VII.0 library is the same as the same thermal
scattering data as the ENDF/B-VII.1 library, with the primary difference being that the thermal
neutron scattering data in ENDF/B-VII.1 uses a continuous representation
that \SerpentTWO v2.1.32 does not support.

To ensure data consistency, I downloaded the \verb,.ace, files from the \Gls{lanl}
website, then processed these files into HDF5 format using the \OpenMC Python
API. I also used the Python API to create a depletion chain from the
spontaneous and delayed fission yield data, decay data, and neutron cross
section data from the ENDF B/VII.1 library\footnote{Interested readers are able to
create this library files by running the scripts located at
\url{https://github.com/arfc/saltproc/tree/master/scripts/xsdata}. See
the README.md in the parent directory for user instructions}.

As discussed in Section \ref{sub:msbr-fuel-salt}, the average temperature of the
fuel between the core inlet and outlets is around 900K, and this is the value I used
for all material temperatures. For cross section data
unavailable at that temperature, I used interpolation between 800K and 1000
to get reasonable values for the cross sections.

\section{Comparison of OpenMC to Serpent}
\label{sec:openmc-vs-serpent}

\begin{figure}[htpb]
    \centering
    \subfloat[][]{
        \includegraphics[width=0.8\linewidth]{figs/ch5/keff.pdf}
        \label{fig:keff}

    }\\
    \subfloat[][]{
        \includegraphics[width=0.8\linewidth]{figs/ch5/keff_error.pdf}
        \label{fig:keff_err}
    }
    \caption[$k_\text{eff}$ difference between \OpenMC and \SerpentTWO over time]{
        \subref{fig:keff} $k_\text{eff}$ difference between \OpenMC and \SerpentTWO over time;
        \subref{fig:keff_err} $k_\text{eff}$ relative difference between \OpenMC and \SerpentTWO 
        over time.
    }
    \label{fig:keff_sum}
\end{figure}

Figure \ref{fig:keff_sum} shows the change in $k_\text{eff}$ over time in the \OpenMC and \SerpentTWO
coupled simulations, as well as the relative difference between.
In Figure \ref{fig:keff}. the error bars are black, and in Figure
\ref{fig:keff_err}, they are green. For the first 70 days or so, the
$k_\text{eff}$ calculated by \OpenMC is slightly  higher than the 
$k_\text{eff}$ calculated by \SerpentTWO, with a difference of around 150 pcm.
After 70 days, the $k_\text{eff}$ calculated by \OpenMC is becomes smaller than
the $k_\text{eff}$ calculated by \SerpentTWO. As time goes on, the difference
between the two grows constantly. At the end of the simulation, the absolute
value of the relative difference is around 0.6\%, or 600 pcm. Notice the
oscillating behavior of $k_\text{eff}$. This is due to fuel reprocessing every 3 days.

% If we did not reprocess the fuel, the simulation
%would look like \ldots

\subsection{Nuclide compositions}
\label{sub:nuclide-compositions}

\begin{figure}[htpb]
    \centering
    \includegraphics[width=0.8\textwidth]{figs/ch5/actinides.pdf}
    \caption[Relative difference of actinides mass in fuel at selected time steps]{Relative difference of actinides mass between \OpenMC and \SerpentTWO
    coupled simulations at 75, 147, 222, and 294 days. \ce{^{242m}Am}, \ce{^{242}Cm},
    \ce{^{245}Cm}, \ce{^{246}Cm} have been ommited here due to their high difference,
    and can be found in Figures \ref{fig:am242m-mass}, \ref{fig:cm242-mass},
    \ref{fig:cm245-mass}, and \ref{fig:cm246-mass}, resepectively.
    \label{fig:actinides}
\end{figure}

Figure \ref{fig:actinides} shows the relative error of actinide mass in the fuel
salt at several points in the simulation. As time goes on, the error for
\ce{^{241}Am}, \ce{^{242}Am}, and \ce{^{235}U} generally decreases in
magnitude, and for the rest of the actinides it increases in magnitude. The
maximum error for the actinides is around 8\% at the end of the simulation.

The main fissionable nuclide in our fuel salt is \ce{^{233}U}, and the
composition in the \OpenMC fuel is less than the composition in the \SerpentTWO
fuel. This is consistent with our results for $k_\text{eff}$.

\begin{figure}[htpb]
    \centering
    \includegraphics[width=0.8\textwidth]{figs/ch5/fission_products.pdf}
    \caption{Relative error of fission products mass in fuel. \ce{^{97}Mo}
        has been omitted to
        maintain clarity in the results. It's error is around -20\%
    }
    \label{fig:fission-products}
\end{figure}

Figure \ref{fig:fission-products} shows the relative error of fission product
mass in the fuel. The behavior of the error varies depending on the nuclide.
The maximum error for the fission products that are not cesium is around 3\%
and the end of the simulation.

%go into detail about each nuclide?

\ce{^{242}Cm} is one of the few actinides with significant error early in the
simulation that even out later on. Figure \ref{fig:cm242-mass} shows this
behavior

\begin{figure}[htpb]
    \centering
    \subfloat[][]{
        \includegraphics[width=0.5\linewidth]{figs/ch5/Cm242_mass_0.pdf}
        \label{fig:cm242-mass-0}
    }
    \subfloat[][]{
        \includegraphics[width=0.5\linewidth]{figs/ch5/Cm242_mass_1.pdf}
        \label{fig:cm242-mass-1}
    }
    \caption{}
    \caption[\ce{^{242}Cm} mass]{
    \subref{fig:cm242-mass-0} \ce{^{242}Cm} mass up to 183 days;
    \subref{fig:cm242-mass-1} \ce{^{242}Cm} mass after 183 days}
    \label{fig:cm242-mass}
\end{figure}

There are several important actinides, that
have very large errors for all but the first depletion step, specifically
\ce{^{242m}Am}, \ce{^{245}Cm}, and  \ce{^{246}Cm}. Figures
\ref{fig:am242_m1-mass}, \ref{fig:cm245-mass}, and \ref{fig:cm246-mass} show
the behavior for each of these nuclides, respectively.

\begin{figure}[htpb]
    \centering
    \subfloat[][]{
        \includegraphics[width=0.5\linewidth]{figs/ch5/Am242_m1_mass_0.pdf}
        \label{fig:am242_m1-mass-0}
    }
    \subfloat[][]{
        \includegraphics[width=0.5\linewidth]{figs/ch5/Am242_m1_mass_1.pdf}
        \label{fig:am242_m1-mass-1}
    }
    \caption[\ce{^{242m}Am} mass]{
    \subref{fig:am242_m1-mass-0} \ce{^{242m}Am} mass up to 183 days;
    \subref{fig:am242_m1-mass-1} \ce{^{242m}Am} mass after 183 days}
    \label{fig:am242_m1-mass}
\end{figure}

\begin{figure}[htpb]
    \centering
    \subfloat[][]{
        \includegraphics[width=0.5\linewidth]{figs/ch5/Cm245_mass_0.pdf}
        \label{fig:cm245-mass-0}
    }
    \subfloat[][]{
        \includegraphics[width=0.5\linewidth]{figs/ch5/Cm245_mass_1.pdf}
        \label{fig:cm245-mass-1}
    }
    \caption{}
    \caption[\ce{^{245}Cm} mass]{
    \subref{fig:cm245-mass-0} \ce{^{245}Cm} mass up to 183 days;
    \subref{fig:cm245-mass-1} \ce{^{245}Cm} mass after 183 days}
    \label{fig:cm245-mass}
\end{figure}

\begin{figure}[htpb]
    \centering
    \subfloat[][]{
        \includegraphics[width=0.5\linewidth]{figs/ch5/Cm246_mass_0.pdf}
        \label{fig:cm246-mass-0}
    }
    \subfloat[][]{
        \includegraphics[width=0.5\linewidth]{figs/ch5/Cm246_mass_1.pdf}
        \label{fig:cm246-mass-1}
    }
    \caption{}
    \caption[\ce{^{246}Cm} mass]{
    \subref{fig:cm246-mass-0} \ce{^{246}Cm} mass up to 183 days;
    \subref{fig:cm246-mass-1} \ce{^{246}Cm} mass after 183 days}
    \label{fig:cm246-mass}
\end{figure}


From a computational perspective, the high errors after the first timestep may
be due to the extremely small amount of these nuclides present in the material.
This would not effect the first depletion step, as the materials for both
simulations have been verified to be using the same initial compositions.
However, due to the control flow described in section
\ref{sec:saltproc-detail}, the nuclide compositions of nuclides present in
small amounts could be effected.

%\subsection{Elements targeted for removal}
%\label{sub:removed-elements}

\chapter{Conclusions}
\label{ch:chapter6}
\glsresetall

In this thesis, I have discussed an approximate \Gls{csg} model of the
\Gls{msbr} for use in \OpenMC. I have demonstrated this model to be consistent
with the \SerpentTWO model it is based on through a convergence study as well as
through demonstration of the results of the new \OpenMC functionality in
\SaltProc in comparison with the \SerpentTWO functionality.
I found that the difference in $k_\text{eff}$ averaged 35 pcm
towards the beginning of the simulation, however this grew to
700 pcm towards the end of the simulation. I attributed this growth
to the difference in \ce{^{232}Th} between the \OpenMC and \SerpentTWO
calculations. I also found that actinides in general has very small
differences in mass between the two codes, around 2\%, and fission products
had even lower differences in mass, around 0.6\%. Finally, I attributed
large numerical discrepancies in the masses of \ce{^{241}Am}, \ce{^{242}Am},
\ce{^{242m}Am}, \ce{^{242}Cm}, \ce{^{245}Cm}, and \ce{^{246}Cm} present in the fuel
to an experimental \OpenMC feature I added to ensure accurate results for decay-only
nuclides, as the numerical discrepancies were only present at the beginning of the simulation.

\section{Future Work}
With the release of v0.5.0, \SaltProc has the capability to run
with a completely open source workflow. Even so, There are several issues facing \SaltProc
that I want to bring attention to. Not all of these are problems with obvious
solutions, but they are still important to consider as they effect the long
term survival of the tool.

First, the need to reinitialize the geometry, neutronics
settings, and cross sections after applying the processing scheme to the
materials is computationally inefficient. The most time consuming of these tasks
is loading cross sections, which can take up several node hours of additional
time if there are many timesteps. The solution to this problem depends on
the presence of an API, which will be depletion code-dependent.

Second, while in principle the tool is
designed to be compatible with any depletion code, in practice, this requires
implementing a new \verb.Depcode. class, which can potentially require major
refactoring as the work for this thesis did.

Third, as an interface code, \SaltProc needs to stay up-to-date with the
software it is intended for use with (\OpenMC and \SerpentTWO). If there are
major changes in the relevant parts of the API or results format, this change
will need to be addressed in a new \SaltProc release. While this is a normal
component of software development, right now there is only a single active
maintainer of \SaltProc (myself), and it is not a guarantee that I will
be able to continue to maintain it in the future.

The addition of continuous reprocessing functionality into \SerpentTWO -- and 
just recently, in  \OpenMC -- also call in to question the necessity for a
batch-wise reprocessing tool for advanced nuclear reactors. In countries using
LWRs that require fuel to be reprocessed (France and Japan, for example) perform
reprocessing, the reprocessing is physically batch-wise, happening once every 6
months or so. In these use cases, \SaltProc is uniquely poised to simulate them,
as it is still the only open-source tool of its kind.


\backmatter

\bibliographystyle{apalike}
\bibliography{bibliography}

\mainmatter
\appendix
\chapter{Appendix A: Derivation of extraction efficiency equation}
\label{appex:extraction-efficiency}

Consider a material containing $m$ grams of element $X$. Suppose we apply
a process to this material, and let the extraction efficiency for element $X$ be
$\epsilon_{X}$. The mass of element $X$ remaining in the material after applying
the process, $m_{1}$ is
\begin{equation}
   m_{1} = m(1-\epsilon_{X}) 
\end{equation}

Suppose now that we apply the same process to the material a second time. The mass of element $X$ remaining in the material, $m_{2}$ is 
\begin{equation}
    m_{2} = m_{1}(1-\epsilon_{X}) = m(1-\epsilon_{X})^{2}
\end{equation}

It follow by induction that the leftover mass of element $X$ after applying this
process $n$ times is
\begin{equation}
    m_{n} = m_(1-\epsilon_{X})^{n}
\end{equation}

Now, let $\delta \equiv \frac{m_{n}}{m}$, the fraction of original mass of
element $X$ remaining in the material. Suppose we pick a process that represents
a real chemical process with a cycle time\footnote{the amount of time it takes
to remove 100\% of the element} of $c_{X}$ for element $X$. Let $t$ be a value
of time between 0 and $c_{X}$. Then, at time $t$, the fraction of mass
of element $X$ remaining in the material is given by

\begin{equation}
    \delta = (1-\epsilon_{X})^{\frac{c_{X}}{t}}
\end{equation}

Consider the case where $t=c_{X}$. There is no value of $\epsilon_{X} < 1$ that
gives $\delta = 0$. It is also very difficult to seprate 100\% of a substance
from a material in chemistry, so a natural solution arises in assinging a small,
nonzero value to $\delta$ as a constraint to obtain a seprataion efficiency.
Rearranging our equation, we get

\begin{equation}
    \epsilon_{X} = 1 - \delta\frac{t}{c_{X}}
\end{equation}

If $t$ is greater than $c_{X}$, then the above equation no longer applies and we simply assign $\epsilon_{X} = 1$





\end{document}
\endinput
%%
%% End of file `thesis-ex.tex'.
