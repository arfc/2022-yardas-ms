%% Package and Class "uiucthesis2014" for use with LaTeX2e.
\documentclass[edeposit,fullpage]{uiucthesis2018}


\usepackage[acronym,toc]{glossaries}
\newacronym{crpg}{CRPG}{Computational Reactor Physics Group}
\newacronym{ornl}{ORNL}{Oak Ridge National Laboratory}
\newacronym{lanl}{LANL}{Los Alamos National Laboratory}
\newacronym{inl}{INL}{Idaho National Laboratory}
\newacronym{mit}{MIT}{Massachusets Institute of Technology}
\newacronym{csg}{CSG}{constructive solid geometry}
\newacronym{cfd}{CFD}{computational fluid dynamics}
\newacronym{cad}{CAD}{computer aided design}
\newacronym{dnp}{DNP}{delayed neutron precursor}
\newacronym{ghg}{GHG}{greenhouse gas}
\newacronym{gif}{GIF}{Generation IV International Forum}
\newacronym{msr}{MSR}{molten salt reactor}
\newacronym{msre}{MSRE}{Molten Salt Reactor Experiment}
\newacronym{msbr}{MSBR}{Molten Salt Breeder Reactor}
\newacronym{msfr}{MSFR}{Molten Sodium Fast Reactor}
\newacronym{nrc}{NRC}{Nuclear Regulatory Commission}
\newacronym{rnd}{R\&D}{research and development}
\newacronym{mns}{M\&S}{modeling and simulation}
\newacronym{ent}{E\&T}{education and training}
\newacronym{doene}{DOE-NE}{Department of Energy Office of Nuclear Energy}
\newacronym{cc}{CC}{closed code}
\newacronym{oss}{OSS}{open source software}
\newacronym{iaea}{IAEA}{International Atomic Energy Agency}
\newacronym{oncore}{ONCORE}{Open-source Nuclear COdes for REactor analysis}
\newacronym{snf}{SNF}{spent nuclear fuel}
\newacronym{sfr}{SFR}{sodium-cooled fast reactor}
\newacronym{doe}{DOE}{Department of Energy}
\newacronym{bol}{BOL}{beginning of life}
\newacronym{eol}{EOL}{end of life}
\newacronym{oop}{OOP}{object oriented programming}
\newacronym{tap}{TAP}{Transatomic Power}


\usepackage{xspace}
\usepackage{graphics}
\newcommand{\Cycamore}{\textsc{Cycamore}\xspace}
\newcommand{\Cyclus}{\textsc{Cyclus}\xspace}


\usepackage{placeins}
\usepackage{booktabs} % nice rules (thick lines) for tables
\usepackage{microtype} % improves typography for PDF

\usepackage[hyphens]{url}
\usepackage{hyperref}
\usepackage{subfig}
\usepackage{hhline}
\usepackage{amsmath}
\usepackage{color}
\usepackage{multirow}
\usepackage{siunitx}
\sisetup{
    input-decimal-markers = .,input-ignore = {,},table-number-alignment = right,
    group-separator={,}, group-four-digits = true
}
\usepackage{fourier}
\usepackage{booktabs}
\newcommand\tab[1][1cm]{\hspace*{#1}}

\usepackage{threeparttable, tablefootnote}

%tikzpicture fit to page width
\usepackage{environ}
\makeatletter
\newsavebox{\measure@tikzpicture}
\NewEnviron{scaletikzpicturetowidth}[1]{%
  \def\tikz@width{#1}%
  \def\tikzscale{1}\begin{lrbox}{\measure@tikzpicture}%
  \BODY
  \end{lrbox}
  \pgfmathparse{#1/\wd\measure@tikzpicture}%
  \edef\tikzscale{\pgfmathresult}%
  \BODY
}

\usepackage{tabularx}
\newcolumntype{b}{>{\hsize=1.0\hsize}X}
\newcolumntype{q}{>{\hsize=0.5\hsize}X}
\newcolumntype{R}{>{\raggedleft\arraybackslash\hsize=0.5\hsize}X}
\newcolumntype{z}{>{\hsize=0.75\hsize}X}
\newcolumntype{s}{>{\hsize=.5\hsize}X}
\newcolumntype{m}{>{\hsize=.75\hsize}X}

\usepackage{cleveref}
\usepackage{datatool}
\usepackage[numbers]{natbib}
\usepackage{notoccite}


\usepackage{tikz}
\usetikzlibrary{positioning, arrows, decorations, shapes}

\usetikzlibrary{shapes.geometric,arrows}
\tikzstyle{process} = [rectangle, rounded corners, minimum width=2.5cm, minimum height=1cm,text centered, draw=black, fill=blue!30]

\tikzstyle{object} = [ellipse, rounded corners, minimum width=3cm, minimum height=1cm,text centered, draw=black, fill=green!30]
\tikzstyle{objectr} = [ellipse, rounded corners, minimum width=3cm, minimum height=1cm,text centered, draw=black, fill=red!30]

\tikzstyle{empty} =  [rectangle, rounded corners, minimum width=2.5cm, minimum height=0.7cm,text centered, draw=black, fill=white!30]
\tikzstyle{arrow} = [thick,->,>=stealth]


\title{Example Template}
\author{Oleksandr Redin Yardas}
\department{Nuclear, Plasma, Radiological Engineering}
\schools{B.A., Grinnell College, 2020}
\msthesis
\advisor{Madicken Munk}
\degreeyear{2022}
\committee{Research Scientist Madicken Munk, Advisor \\ ... }


\begin{document}
\maketitle

\frontmatter
%% Create an abstract that can also be used for the ProQuest abstract.
%% Note that ProQuest truncates their abstracts at 350 words.
\begin{abstract}

Abstract.

\end{abstract}

\chapter*{Acknowledgments}

Acks.

%% The thesis format requires the Table of Contents to come
%% before any other major sections, all of these sections after
%% the Table of Contents must be listed therein (i.e., use \chapter,
%% not \chapter*).  Common sections to have between the Table of
%% Contents and the main text are:
%%
%% List of Tables
%% List of Figures
%% List Symbols and/or Abbreviations
%% etc.

\tableofcontents
\listoftables
\listoffigures

%% Create a List of Abbreviations. The left column
%% is 1 inch wide and left-justified
%\chapter{List of Abbreviations}
%\printglossaries
%% Create a List of Symbols. The left column
%% is 0.7 inch wide and centered

\pagebreak
\mainmatter

\chapter{Introduction}
Introduction \cite{huff_extensions_2014}.

\chapter*{Appendix}
\appendix
\chapter*{Appendix B: SaltProc input file structure}
\label{appex:input-files}
\definecolor{bg}{rgb}{0.95,0.95,0.95}

\begin{listing}[!ht]
    \begin{minted}[frame=single,
                   framesep=3mm,
                   tabsize=4,
                   bgcolor=bg,
                   fontsize=\footnotesize]{json}
    {
      "Path to Serpent executable": "sss2",
      "File containing processing system objects": "msbr_objects.json",
      "Graph file containing processing system structure": "msbr.dot",
      "User's Serpent input file with reactor model": "msbr.serpent",
      "Path output data storing folder": "../../saltproc/data/",
      "Output HDF5 database file name": "msbr_kl_100_saltproc.h5",
      "Number of neutrons per generation": 50,
      "Number of active generations": 20,
      "Number of inactive generations": 20,
      "Restart simulation from the step when it stopped?": false,
      "Geometry file/files to use in Serpent runs": "geometry/msbr_full.ini",
      "Switch to another geometry when keff drops below 1?": false,
      "Salt mass flow rate throughout reactor core (g/s)": 9920000,
      "Number of steps for constant power and depletion interval case": 12,
      "Depletion step interval or Cumulative time (end of step) (d)": 3,
      "Reactor power or power step list during depletion step (W)": 2250000000
    }
    \end{minted}
    \caption{\SaltProc v0.3.0 input file}
    \label{listing:1}
\end{listing}

\begin{listing}[!ht]
    \begin{minted}[frame=single,
                   framesep=3mm,
                   tabsize=4,
                   bgcolor=bg,
                   fontsize=\footnotesize]{json}
    {
       "proc_input_file": "msbr_objects.json",
       "dot_input_file": "msbr.dot",
       "output_path": "./data",
       "num_depsteps": 12,
       "depcode": {
           "codename": "serpent",
           "exec_path": "sss2",
           "template_inputfile_path": "./msbr.serpent",
           "iter_inputfile": "saltproc_serpent",
           "iter_matfile": "saltproc_mat",
           "npop": 50,
           "active_cycles": 20,
           "inactive_cycles": 20,
           "geo_file_paths": ["./geometry/msbr_full.ini"]
       },
       "simulation": {
           "sim_name": "msbr_example_simulation",
           "db_name": "msbr_kl_100_saltproc.h5",
           "restart_flag": false,
           "adjust_geo": false
       },
       "reactor": {
           "volume": 1.0,
           "mass_flowrate": 9920000,
           "power_levels": [ 2250000000 ],
           "dep_step_length_cumulative": [ 3 ]
       }
    }
    \end{minted}
    \caption{\SaltProc v0.4.0 input file}
    \label{listing:2}
\end{listing}

\begin{listing}[!ht]
    \begin{minted}[frame=single,
                   framesep=3mm,
                   tabsize=4,
                   bgcolor=bg,
                   fontsize=\footnotesize]{json}
    {
       "proc_input_file": "msbr_objects.json",
       "dot_input_file": "msbr.dot",
       "n_depletion_steps": 12,
       "depcode": {
           "codename": "serpent",
           "template_input_file_path": "msbr.serpent",
           "geo_file_paths": ["geometry/msbr_full.ini"]
       },
       "simulation": {
           "sim_name": "msbr_kl_100_simulation"
       },
       "reactor": {
           "volume": 1.0,
           "mass_flowrate": 9920000,
           "power_levels": [ 2250000000 ],
           "depletion_timesteps": [ 3 ],
           "timestep_units": "d"
       }
    }
    \end{minted}
    \caption{\SaltProc v0.5.0 input file}
    \label{listing:3}
\end{listing}




\backmatter

\bibliographystyle{apalike}
\bibliography{bibliography}

\end{document}
\endinput
%%
%% End of file `thesis-ex.tex'.
